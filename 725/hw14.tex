\documentclass[11pt]{amsart}
\usepackage{amssymb,amsmath,latexsym,enumerate,mathptmx,microtype}
\hoffset=0in 
\voffset=0in
\oddsidemargin=0in
\evensidemargin=0in
\topmargin=-.2in 
\textwidth=6.5in
\textheight=9in
\begin{document}
\setlength{\parindent}{0pt}
\setlength{\parskip}{0.4cm}
\thispagestyle{empty} 
\def\Id{\mathrm I}
\def\0{\mathbf 0}
\def\b{\mathbf b}
\def\c{\mathbf c}
\def\e{\mathbf e}
\def\f{\mathbf f}
\def\u{\mathbf u}
\def\v{\mathbf v}
\def\w{\mathbf w}
\def\x{\mathbf x}
\def\y{\mathbf y}
\def\C{\mathbf{C}}
\def\F{\mathbf{F}}
\def\R{\mathbf{R}}
\def\Z{\mathbf{Z}}
\def\P{\mathcal{P}}
\newcommand\spn{\operatorname{span}}
\renewcommand\null{\operatorname{null}}
\newcommand\range{\operatorname{range}}
\newcommand\rank{\operatorname{rank}}
\newcommand\mult{\operatorname{mult}}
\newcommand\tr{\operatorname{tr}}
\newcommand\norm[1]{\left|\left| #1 \right|\right|}
\newcommand\inner[2]{\left< #1, #2 \right>}

\begin{center} {\bf MATH 725 \qquad \qquad Homework Set 14 \qquad \qquad due 12/12/11} \end{center} 

\begin{enumerate}[(1)]

\vspace{12pt}

\item Suppose $V$ is a complex inner-product space and $\v, \w \in V$.
Define the linear operator $T(\u) := \inner \u \v \w$. Find a formula for $\tr(T)$.

\begin{proof}[Solution]
Fix an orthonormal basis $\left( \e_1, \e_2, \dots, \e_n \right)$ of $V$, and express $\v$ and $\w$ in terms of this basis:
\[
  \v = \sum_{ j=1 }^n a_j \, \e_j \, ,
  \qquad
  \w = \sum_{ j=1 }^n b_j \, \e_j \, ,
\]
where $a_j := \inner \v {\e_j}$ and $b_j := \inner \w {\e_j}$.
Now
\[
  T(\e_k)
  = \inner {\e_k} \v \w
  = \overline{ a_k } \sum_{ j=1 }^n b_j \, \e_j
\]
and so the $(j,k)$-entry of the matrix of $T$ (with respect to our basis) is $\overline{ a_k } b_j$.
Thus
\[
  \tr(T)
  = \sum_{ j=1 }^n \overline{ a_j } b_j
  = \inner \w \v . \qedhere
\]
\end{proof}

\item Suppose $V$ is a complex inner-product space, and $T \in L(V)$.
  \begin{enumerate}
  \item Prove that if $\left( \e_1, \e_2, \dots, \e_n \right)$ is an orthonormal basis of $V$ then
  \[
    \tr \left( T^* T \right) = \norm{ T (\e_1) }^2 + \norm{ T (\e_2) }^2 + \dots + \norm{ T (\e_n) }^2 .
  \]
  \item Show that if $T$ is positive and $\tr(T) = 0$ then $T=0$, the zero operator.
  \end{enumerate}

\begin{proof}
\begin{enumerate}

\item By definition,
\[
  \tr \left( T^* T \right)
   = \sum_{ j=1 }^n \inner{ T^* T(\e_j) }{ \e_j }
   = \sum_{ j=1 }^n \inner{ T(\e_j) }{ T(\e_j) }
   = \sum_{ j=1 }^n \norm{ T(\e_j) }^2 .
\]

\item Suppose $T$ is positive and $\tr(T) = 0$.
We proved in class (some time ago) that, since $T$ is positive, there exists $S \in L(V)$ such that $T = S^* S$. Thus, by part (a),
\[
  0 = \tr(T)
    = \tr(S^* S)
    = \sum_{ j=1 }^n \norm{ S(\e_j) }^2 ,
\]
and so $S(\e_j) = 0$ for $j = 1, \dots, n$. But then $S$ is the zero operator and thus $T = 0$. \qedhere

\end{enumerate}
\end{proof}

\item Construct counterexamples to the following claims about $S, T \in L(V)$:
  \begin{enumerate}
  \item $\tr(ST) = \tr(S) \tr(T)$.
  \item $\det(S+T) = \det(S) + \det(T)$.
  \end{enumerate}

\begin{proof}[Solution]
\begin{enumerate}

\item Let $S = T = \begin{bmatrix} 0&1\\ 1&0 \end{bmatrix}$ (with respect to the standard basis of $\C^2$). Then $\tr(S) = \tr(T) = 0$ but $\tr(ST) = \tr(\Id) = 2$.

\item Let $S$ and $T$ be the identity operator on $\C^2$. Then $S+T = \begin{bmatrix} 2&0\\ 0&2 \end{bmatrix}$ with determinant 4, whereas $\det(S) = \det(T) = 1$. \qedhere

\end{enumerate}
\end{proof}

\item Suppose $V$ is a complex inner-product space.
  \begin{enumerate}
  \item Prove that $\inner S T := \tr \left( S \, T^* \right)$ defines an inner product on $L(V)$.
  \item Fix an orthonormal basis of $V$ and let $\left( a_{ jk } \right)_{ 1 \le j, k \le n }$ be the matrix of $T \in L(V)$ with respect to this basis. Show that
  \[
    \inner T T = \sum_{ 1 \le j, k \le n } \left| a_{ jk } \right|^2 ,
  \]
  the square of the standard norm on $L(V)$ (identifying operators with their matrices with respect to the chosen orthonormal basis).
  Conclude that $\inner S T$ coincides with the standard inner product on~$L(V)$.
  \end{enumerate}

\begin{proof}
\begin{enumerate}

\item $\inner T T \ge 0$ and $\inner T T = 0 \ \Longleftrightarrow \ T = 0$ follow with Exercise 2(a).
Second,
\begin{align*}
  \inner { aR+S } T
  &= \tr \left( (aR+S) T^* \right)
   = \tr \left( aR T^* + S T^* \right)
   = \tr \left( aR T^* \right) + \tr \left( S T^* \right) \\
  &= a \tr \left( R T^* \right) + \tr \left( S T^* \right)
   = a \tr \inner R {T^*} + \inner S {T^*} \, .
\end{align*}
Here we have used that for any linear operator $T$ and $a \in \C$, $\tr(aT) = a \tr(T)$, which is easily seen to be true by writing $T$ in terms of a matrix.

For the final property that will determine that $\inner S T$ is indeed a inner product, we will need that for any $T \in L(V)$,
\[
  \tr(T^*)
   = \sum_{ j=1 }^n \inner{ T^*(\e_j) }{ \e_j }
   = \sum_{ j=1 }^n \inner{ \e_j }{ T(\e_j) }
   = \sum_{ j=1 }^n \overline{ \inner{ T(\e_j) }{ \e_j } }
   = \overline{ \sum_{ j=1 }^n \inner{ T(\e_j) }{ \e_j } }
   = \overline{ \tr(T) } \, .
\]
Thus
\[
  \inner S T
  = \tr(ST^*)
  = \overline{ \tr(TS^*) }
  = \overline{ \inner T S } \, ,
\]
and thus $\inner S T$ satisfies all the properties of an inner product.

\item By Exercise 2(a),
\[
  \inner T T
  = \tr(TT^*)
  = \tr(T^* T)
  = \sum_{ k=1 }^n \norm{ T(\e_k) }^2
  = \sum_{ k=1 }^n \sum_{ j=1 }^n \left| a_{ jk } \right|^2 .
\]
As the statement of the exercise says, this is the square of the standard norm on $L(V)$.
Thus (by Exercise 2 of Homework Set 7) this norm induces the standard inner product on~$L(V)$. \qedhere

\end{enumerate}
\end{proof}

\item Suppose $V$ is a complex inner-product space and $T \in L(V)$.
  \begin{enumerate}
  \item Show that $\det(T^*) = \overline{ \det(T) }$.
  \item Show that $\left| \det(T) \right| = \det \sqrt{ T^* T }$.
  \end{enumerate}

\begin{proof}
\begin{enumerate}

\item Let $\lambda_1, \lambda_2, \dots, \lambda_n$ be the eigenvalues of $T$ (listed with repetition according to multiplicities). We proved earlier that $\overline{\lambda_1}, \overline{\lambda_2}, \dots, \overline{\lambda_n}$ are then the eigenvalues of $T^*$, and so
\[
  \det(T^*)
  = \overline{\lambda_1} \, \overline{\lambda_2} \cdots \overline{\lambda_n}
  = \overline{ \lambda_1 \, \lambda_2 \cdots \lambda_n }
  = \overline{ \det(T) } \, .
\]

\item We know that $T = S \sqrt{ T^* T }$ for some isometry $S$.
Note that $|\det(S)| = 1$ and $\sqrt{ T^* T }$ is positive (and thus has a nonnegative determinant), and so
\[
  |\det(T)|
  = \left| \det(S) \det \sqrt{ T^* T } \right|
  = \det \sqrt{ T^* T } \, . \qedhere
\]

\end{enumerate}
\end{proof}

\end{enumerate}

\end{document}