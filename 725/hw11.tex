\documentclass[11pt]{amsart}
\usepackage{amssymb,amsmath,latexsym,enumerate,mathptmx,microtype}
\hoffset=0in 
\voffset=0in
\oddsidemargin=0in
\evensidemargin=0in
\topmargin=-.2in 
\textwidth=6.5in
\textheight=9in
\begin{document}
\setlength{\parindent}{0pt}
\setlength{\parskip}{0.4cm}
\thispagestyle{empty} 
\def\Id{\mathrm I}
\def\0{\mathbf 0}
\def\b{\mathbf b}
\def\c{\mathbf c}
\def\e{\mathbf e}
\def\f{\mathbf f}
\def\u{\mathbf u}
\def\v{\mathbf v}
\def\w{\mathbf w}
\def\x{\mathbf x}
\def\y{\mathbf y}
\def\C{\mathbf{C}}
\def\F{\mathbf{F}}
\def\R{\mathbf{R}}
\def\Z{\mathbf{Z}}
\def\P{\mathcal{P}}
\newcommand\spn{\operatorname{span}}
\renewcommand\null{\operatorname{null}}
\newcommand\range{\operatorname{range}}
\newcommand\rank{\operatorname{rank}}
\newcommand\mult{\operatorname{mult}}
\newcommand\norm[1]{\left|\left| #1 \right|\right|}
\newcommand\inner[2]{\left< #1, #2 \right>}

\begin{center} {\bf MATH 725 \qquad \qquad Homework Set 11 \qquad \qquad due 11/14/11} \end{center} 

\begin{enumerate}[(1)]

\vspace{12pt}

\item Compute the singular values of the differentiation operator on $\P_2(\R)$ equipped with our usual inner product $\inner f g := \int_{ -1 }^1 f(x) \overline{g(x)} \, dx \, .$

\begin{proof}[Solution]
We computed (in an earlier homework) the orthonormal basis
\[
  \e_1 = \tfrac{ 1 }{ \sqrt 2 } \, \qquad
  \e_2 = \sqrt{ \tfrac 3 2 } \, x \, \qquad
  \e_3 = \sqrt{ \tfrac{ 45 }{ 8 }  }  \left( x^2 - \tfrac{ 1 }{ 3 } \right)
\]
for this inner-product space. With respect to this basis, the differentiation operator $D$ has the matrix
\[
  D = \left( \begin{array}{ccc}
  0 & \sqrt 3 & 0 \\
  0 & 0 & \sqrt{15} \\
  0 & 0 & 0
  \end{array} \right) .
\]
Since our basis is orthonormal, we can compute $D^*$ by transposing this matrix:
\[
  D^*  = \left( \begin{array}{ccc}
  0 & 0 & 0 \\
  \sqrt 3 & 0 & 0 \\
  0 & \sqrt{15} & 0
  \end{array} \right) ,
\]
and so
\[
  \sqrt{ D^* D } = \left( \begin{array}{ccc}
  0 & 0 & 0 \\
  0 & \sqrt 3 & 0 \\
  0 & 0 & \sqrt{15}
  \end{array} \right) .
\]
Thus the singular values of $D$ (which are the eigenvalues of $\sqrt{ D^* D }$) are 0, $\sqrt 3$, and $\sqrt{15}$.
\end{proof}

\item Consider $T \in L(\C^3)$ given by $T(x,y,z) = (y,z,0)$.
  \begin{enumerate}
  \item Compute the eigenvalues and generalized eigenspaces of $T$.
  \item Prove that $T$ has no square root.\footnote{\emph{Hint:} start by showing that if $S^2 = T$ then $S^3 = 0$.}
  \end{enumerate}

\begin{proof}
\begin{enumerate}

\item
Writing $T$ in terms of the standard basis,
\[
  T = \left( \begin{array}{ccc}
  0 & 1 & 0 \\
  0 & 0 & 1 \\
  0 & 0 & 0
  \end{array} \right) ,
\]
we see that $\dim \null(T^2) = 2$ and $\dim \null(T^3) = 3$, and so 0 is the only eigenvalue of $T$ and it has multiplicity 3, i.e., the corresponding generalized eigenspace is all of~$\C^3$.

\item
Suppose $S \in L(\C^3)$ is a square root of $T$, i.e., $S^2 = T$.
By part (a), $\dim \null (S^6) = \dim \null(T^3) = 3$, and so (by a proposition proved in class)
\[
  \dim \null(S^3) = \dim \null(S^4) = \dim \null(S^5) = \dim \null (S^6) = 3 \, .
\]
But this means that $\dim \null(T^2) = \dim \null(S^4) = 3$, contrary to what we showed in part~(a). \qedhere

\end{enumerate}
\end{proof}

\item Suppose $T \in L(V)$ and $m \in \Z_{ \ge 0 }$ such that $\range(T^m) = \range(T^{ m+1 })$. Prove that $\range(T^k) = \range(T^m)$ for all $k \ge m$. 

\begin{proof}
Suppose that $\range(T^m) = \range(T^{ m+1 })$. We will prove that $\range(T^{ m+1 }) = \range(T^{ m+2 })$; the general statement will then follow by induction.

We already know that $\range(T^{ m+2 }) \subseteq \range(T^{ m+1 })$.
To prove the other inclusion, assume that $\v \in \range(T^{ m+1 })$, i.e., $T^{ m+1 } (\u) = \v$ for some $\u \in V$.
Let $\w := T^m(\u)$; be definition $\w \in \range(T^m)$.
By the first line of our proof, $\w \in \range(T^{ m+1 })$, and so $T^{ m+1 } (\x) = \w$ for some $\x \in V$.
Thus
\[
  \v = T^{ m+1 } (\u) = T(\w) = T^{ m+2 } (\x) \, ,
\]
which implies that $\v \in \range(T^{ m+2 })$.
\end{proof}

\item Suppose $T \in L(V)$ where $\dim V = n$. Show that, while in general it is not true that $V = \null(T) \oplus \range(T)$, we always have
\[
  V = \null(T^n) \oplus \range(T^n) \, .
\]

\begin{proof}
The example $V = \R^2$, $T(x,y) = (y,0)$ shows that $V = \null(T) \oplus \range(T)$ is not always true (in this example, $\null(T) = \range(T) = \left\{ (x,0) : \, x \in \R \right\}$).

Next we'll prove that $V = \null(T^n) + \range(T^n)$. Given $\v \in V$, we first remember that we proved in class $\range(T^n) = \range(T^{ 2n })$, and so $T^n(\v)$ (which is by definition in $\range(T^n)$) is in $\range(T^{ 2n })$, i.e.,
\[
  T^n(\v) = T^{ 2n } (\u)
\]
for some $\u \in V$. But this can be rewritten as $T^n \left( \v - T^n(\u) \right) = \0$, i.e., $\v - T^n(\u) \in \null(T^n)$, i.e., $v \in \null(T^n) + \range(T^n)$.

Finally, we'll prove that $\null(T^n) + \range(T^n)$ is a direct sum.
Suppose $\v \in \null(T^n) \cap \range(T^n)$, i.e.,
\[
  T^n (\v) = \0
  \qquad \text{ and } \qquad
  T^n(\u) = \v
\]
for some $\u \in V$. But then $T^{ 2n } (\u) = T^n(\v) = \0$, i.e., $\u \in \null(T^{ 2n }) = \null(T^n)$ (using the same result from class again). This means $\v = T^n(\u) = \0$, and so $\null(T^n) \cap \range(T^n) = \{ \0 \}$ and $V = \null(T^n) \oplus \range(T^n)$.
\end{proof}

\item Suppose $T \in L(V)$ is invertible. Prove that $T$ and $T^{ -1 }$ have the same generalized eigenspaces (even though their eigenvalues are different).
What does this imply for the characteristic polynomials of $T$ and $T^{ -1 }$?

\begin{proof}
Let $n = \dim V$, and consider an eigenvalue $\lambda$ of $T$. Then
\[
  \null (T - \lambda \Id)^n
  = \null \left( \lambda T \left( \tfrac 1 \lambda \Id - T^{ -1 } \right) \right)^n
  = \null \left( (\lambda T)^n \left( \tfrac 1 \lambda \Id - T^{ -1 } \right)^n \right)
\]
for some $j \in \Z_{ >0 } $.
(Here the last equality holds because $T$ commutes with both $\Id$ and $T^{ -1 }$.)
But since $\lambda T$ is invertible,
\[
  \null \left( (\lambda T)^n \left( \tfrac 1 \lambda \Id - T^{ -1 } \right)^n \right)
  = \null \left( \tfrac 1 \lambda \Id - T^{ -1 } \right)^n ,
\]
and so we have
\[
  \null (T - \lambda \Id)^n
  = \null \left( \tfrac 1 \lambda \Id - T^{ -1 } \right)^n ,
\]
in words: the generalized eigenspace of $T$ corresponding to $\lambda$ equals the generalized eigenspace of $T^{ -1 }$ corresponding to $\frac 1 \lambda$.

Let's write $\mult_T(\lambda)$ for the multiplicity of the eigenvalue $\lambda$ of $T$.
What we just proved implies
\[
  \mult_T(\lambda) = \mult_{ T^{-1} } \left( \tfrac 1 \lambda \right) ,
\]
and this implies, in turn, that the characteristic polynomial of $T$,
\[
  c_T (x) := \prod_{ \lambda \text{ eigenvector of } T } (x-\lambda)^{ \mult_T(\lambda) } ,
\]
is related to the following evaluation of the characteristic polynomial of $T^{ -1 }$:
\[
  c_{ T^{ -1 } } \left( \tfrac 1 x \right)
  = \prod_{ \lambda \text{ eigenvector of } T^{ -1 } } \left( \tfrac 1 x - \lambda \right)^{ \mult_{ T^{ -1 } }(\lambda) }
  = \prod_{ \lambda \text{ eigenvector of } T } \left( \tfrac 1 x - \tfrac 1 \lambda \right)^{ \mult_{ T }(\lambda) } .
\]
More precisely, $x^n \, c_{ T^{ -1 } } \left( \tfrac 1 x \right)$ is a polynomial in $x$ of degree $n$, which has the same roots (including multiplicities) as the polynomial $c_T(x)$.
Thus these two polynomials only differ in a constant factor; e.g., we can state their relationship as
\[
  c_T(0) \, x^n \, c_{ T^{ -1 } } \left( \tfrac 1 x \right) = c_T(x) \, . \qedhere
\]
\end{proof}

\end{enumerate}

\end{document}