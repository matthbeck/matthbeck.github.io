\documentclass[11pt]{amsart}
\usepackage{amssymb,amsmath,latexsym,enumerate,mathptmx,microtype}
\hoffset=0in 
\voffset=0in
\oddsidemargin=0in
\evensidemargin=0in
\topmargin=-.2in 
\textwidth=6.5in
\textheight=9in
\begin{document}
\setlength{\parindent}{0pt}
\setlength{\parskip}{0.4cm}
\thispagestyle{empty} 
\def\Id{\mathrm I}
\def\0{\mathbf 0}
\def\b{\mathbf b}
\def\c{\mathbf c}
\def\e{\mathbf e}
\def\u{\mathbf u}
\def\v{\mathbf v}
\def\w{\mathbf w}
\def\x{\mathbf x}
\def\y{\mathbf y}
\def\C{\mathbf{C}}
\def\F{\mathbf{F}}
\def\R{\mathbf{R}}
\def\Z{\mathbf{Z}}
\def\P{\mathcal{P}}
\newcommand\spn{\operatorname{span}}
\renewcommand\null{\operatorname{null}}
\newcommand\range{\operatorname{range}}
\newcommand\rank{\operatorname{rank}}
\newcommand\norm[1]{\left|\left| #1 \right|\right|}
\newcommand\inner[2]{\left< #1, #2 \right>}

\begin{center} {\bf MATH 725 \qquad \qquad Homework Set 9 \qquad \qquad due 10/31/11} \end{center} 

\begin{enumerate}[(1)]

\item A linear operator $T \in L(V)$ is \emph{unitary} if $T^* T$ is the identity map.
Show that all eigenvalues of a unitary operator have absolute value~1.

\begin{proof}
Suppose $T$ is a unitary operator and $T(\v) = \lambda \v$ for some $\v \ne \0$. Then
\[
  \inner \v \v
  = \inner {T^* T(\v)} \v
  = \inner {T(\v)} {T(\v)}
  = \inner {\lambda \v} {\lambda \v}
  = |\lambda|^2 \inner \v \v ,
\]
and so (since $\inner \v \v \ne 0$) $|\lambda| = 1$.
\end{proof}

\item Suppose $S, T \in L(V)$ are self adjoint.
  \begin{enumerate}
  \item Give an example that shows that $ST$ might not be self adjoint.
  \item Prove that $ST$ is self adjoint if and only if $ST = TS$.
  \end{enumerate}

\begin{proof}[Solution]
\begin{enumerate}

\item Let $S = \begin{bmatrix} 1&0\\ 0&0 \end{bmatrix}$ and $T = \begin{bmatrix} 0&1\\ 1&0 \end{bmatrix}$, both expressed in terms of the standard basis of $\R^2$.
Then $ST = \begin{bmatrix} 0&1\\ 0&0 \end{bmatrix}$ is not self adjoint.

\item Since $S$ and $T$ are self adjoint, we have $(ST)^* = T^* S^* = TS$.
Thus $ST$ is self adjoint if and only if $ST = TS$. \qedhere

\end{enumerate}
\end{proof}

\item If $T \in L(V)$ is normal and $k \in \Z_{ >0 }$ then
\[
  \null \left( T^k \right) = \null \left( T \right)
  \qquad \text{ and } \qquad
  \range \left( T^k \right) = \range \left( T \right) .
\]
\emph{Hint:} for one set inclusion, assuming $\v \in \null \left( T^k \right)$, consider $\inner{ T^* T^{ k-1 } (\v) }{ T^* T^{ k-1 } (\v) }$.

\begin{proof}
Suppose $\v \in \null \left( T^k \right)$. Then
\begin{align*}
  0 
  &= \norm{ T^k(\v) }^2
  = \inner{ T^k(\v) }{ T^k(\v) }
  = \inner{ T^{k-1}(\v) }{ T^* T^k(\v) }
  = \inner{ T^{k-1}(\v) }{ T T^* T^{k-1}(\v) } \\
  &= \inner{ T^* T^{ k-1 } (\v) }{ T^* T^{ k-1 } (\v) } ,
\end{align*}
which implies that $T^* T^{ k-1 } (\v) = \0$. But then
\[
  0 = \inner{ T^* T^{ k-1 } (\v) }{ T^{ k-2 } (\v) }
  = \inner{ T^{ k-1 } (\v) }{ T^{ k-1 } (\v) } ,
\]
which implies, in turn, that $T^{ k-1 } (\v) = \0$, i.e., $\v \in \null \left( T^{ k-1 } \right)$.
We can repeat this proof to conclude that $\v \in \null \left( T^{ k-2 } \right)$, $\v \in \null \left( T^{ k-3 } \right)$, etc., and after a finite number of steps we conclude that $\v \in \null(T)$. This proves $\null \left( T^{ k-1 } \right) \subseteq \null(T)$.

Conversely, suppose $\v \in \null(T)$. Then $T^k(\v) = T^{ k-1 } (T(\v)) = T^{ k-1 } (\0) = \0$. This proves $\null(T) \subseteq \null \left( T^{ k-1 } \right)$.

To see that $\range \left( T^k \right) = \range \left( T \right)$, we note that $\range \left( T^k \right) \subseteq \range \left( T \right)$ (since $\v \in \range \left( T^k \right)$ means that there exists $\u \in V$ such that $T^k(\u) = \v$; but then $T\left( T^{ k-1 } (\u) \right) = \v$, i.e., $\v \in \range(T)$).
But these two subspaces have the same dimension:
\[
  \dim \range \left( T^k \right)
  = \dim V - \dim \null \left( T^k \right)
  = \dim V - \dim \null(T)
  = \dim \range(T) \, ,
\]
and thus $\range \left( T^k \right) = \range \left( T \right)$.
\end{proof}

\item Let $V$ be a complex vector space. Prove that a normal operator in $L(V)$ is self adjoint if and only if all its eigenvalues are real.

\begin{proof}
We proved in class that all eigenvalues of a self-adjoint operator are real.

Conversely, suppose all eigenvalues of $T \in L(V)$ are real. By the (complex version of the) spectral theorem, there exists an orthonormal basis of eigenvectors of $T$, and with respect to this basis, $T$ has a diagonal matrix whose diagonal entries are the eigenvalues. But since these entries are real, $T^* = T$ (neither conjugation nor transposing changes the matrix), i.e., $T$ is self adjoint.
\end{proof}

\item Prove that if $T \in L(V)$ is positive, then so is $T^k$, for any $k \in \Z_{ >0 }$.

\begin{proof}
Suppose $T \in L(V)$ is positive. Note that this implies automatically that $T$ is self adjoint.
We will prove that $T^k$ is positive by (strong) induction on $k \in \Z_{ >0 }$.
The base case $k=1$ is given, and for $k=2$ we have for any $\v \in V$
\[
  \inner{ T^2(\v) }{ \v }
  = \inner{ T(\v) }{ T(\v) }
  \ge 0 \, .
\]
For the induction step, assume that $k \ge 3$ and $T^k$ is positive, i.e., $\inner{ T^k(\v) }{ \v } \ge 0$ for any $\v \in V$. Then for any $\v \in V$
\[
  \inner{ T^{ k+1 } (\v) }{ \v }
  = \inner{ T^k(\v) }{ T(\v) }
  = \inner{ T^{ k-1 } \left( T(\v) \right) }{ T(\v) }
  \ge 0
\]
because $T^{ k-1 }$ is positive by induction hypothesis.
\end{proof}

\end{enumerate}

\end{document}