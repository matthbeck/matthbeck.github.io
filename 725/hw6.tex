\documentclass[11pt]{amsart}
\usepackage{amssymb,amsmath,latexsym,enumerate,mathptmx,microtype}
\hoffset=0in 
\voffset=0in
\oddsidemargin=0in
\evensidemargin=0in
\topmargin=-.7in 
\textwidth=6.5in
\textheight=9.5in
\begin{document}
\setlength{\parindent}{0pt}
\setlength{\parskip}{0.4cm}
\thispagestyle{empty} 
\def\Id{\mathrm I}
\def\0{\mathbf 0}
\def\u{\mathbf u}
\def\v{\mathbf v}
\def\w{\mathbf w}
\def\C{\mathbf{C}}
\def\F{\mathbf{F}}
\def\R{\mathbf{R}}
\def\Z{\mathbf{Z}}
\def\P{\mathcal{P}}
\newcommand\spn{\operatorname{span}}
\renewcommand\null{\operatorname{null}}
\newcommand\range{\operatorname{range}}
\newcommand\rank{\operatorname{rank}}

\begin{center} {\bf MATH 725 \qquad \qquad Homework Set 6 \qquad \qquad due 10/3/11} \end{center} 

\begin{enumerate}[(1)]

\item Define the operator $T: \P_3(\R) \to \P_3(\R)$ by $T(p)(x) = \frac{ d^2 }{ dx^2 } \left( \left( 1+x+x^2 \right) p(x) \right)$.
  \begin{enumerate}
  \item Show that $T$ is linear.
  \item Find all eigenvalues of $T$.
  \item Is $T$ invertible?
  \end{enumerate}

\begin{proof}
\begin{enumerate}

\item Let $p(x), q(x) \in \P_3(\R)$ and $a \in \R$. Then
\begin{align*}
  \frac{ d^2 }{ dx^2 } \left( \left( 1+x+x^2 \right) (a \, p(x) + q(x)) \right)
  &= \frac{ d^2 }{ dx^2 } \left( a \left( 1+x+x^2 \right) p(x) + \left( 1+x+x^2 \right) q(x) \right) \\
  &= a \frac{ d^2 }{ dx^2 } \left( \left( 1+x+x^2 \right) p(x) \right) + \frac{ d^2 }{ dx^2 } \left( \left( 1+x+x^2 \right) q(x) \right) .
\end{align*}

\item Fix the standard basis of $\P_3(\R)$. The images of these basis vectors under $T$ are
\begin{align*}
  T(1) &= \frac{ d^2 }{ dx^2 } \left( 1+x+x^2 \right) = 2 \\
  T(x) &= \frac{ d^2 }{ dx^2 } \left( x+x^2 + x^3 \right) = 2 + 6x \\
  T(x^2) &= \frac{ d^2 }{ dx^2 } \left( x^2 + x^3 + x^4 \right) = 2 + 6x + 12 x^2 \\
  T(x^3) &= \frac{ d^2 }{ dx^2 } \left( x^3 + x^4 + x^5 \right) = 6x + 12 x^2 + 20 x^3
\end{align*}
and so the matrix representation of $T$ with respect to this basis is upper triangular, with the eigenvalues 1, 6, 12, and 20 (which we can read off the diagonal).

\item We found a matrix representation of $T$ that is upper triangular with no zero on the diagonal, so by a theorem proved in class $T$ is invertible. \qedhere

\end{enumerate}
\end{proof}

\item Suppose $V$ is finite dimensional and $S, T \in L(V)$.
  \begin{enumerate}
  \item Prove that $ST$ and $TS$ have the same eigenvalues.
  \item Show that (a) is false if $V$ is infinite dimensional.
  \end{enumerate}

\begin{proof}
\begin{enumerate}

\item If $\lambda \in \F$ is an eigenvalue of $ST$ (with eigenvector $\v$) then $S(T(\v)) = \lambda \, \v$.
Thus
\[
  T(S(T(\v))) = \lambda T(\v) \, ,
\]
and so $\lambda$ is also an eigenvalue of $TS$ (with eigenvector $T(\v)$), unless $T(\v) = \0$.

So we still need to deal with the case $T(\v) = \0$, which means that $\null(T) \ne \{\0\}$. In this case, $\lambda = 0$ (because $\0 = S(T(\v)) = \lambda \, \v$ and $\v \ne \0$).
Since $V$ is finite dimensional,
\[
  \dim V = \dim \null(T) + \rank(T) = \dim \null(TS) + \rank(TS) \, .
\]
Because $\range(TS) \subseteq \range(T)$, we conclude that $\dim \null(TS) \ge \dim \null(T) > 0$, so $\null(TS)$ is nontrivial, which means that $TS$ has eigenvalue $\lambda = 0$.

Reversing the roles of $S$ and $T$ shows the other implication.

\item Let $V = \R^\infty$, the vector space of all sequences in $\R$, and define the linear operators
\[
  S \left( x_1, x_2, x_3, \dots \right) = \left( x_2, x_3, x_4, \dots \right) 
  \qquad \text{ and } \qquad
  T \left( x_1, x_2, x_3, \dots \right) = \left( 0, x_1, x_2, \dots \right) .
\]
Then $ST$ is the identity map (which has 1 as its only eigenvalue) and
\[
  TS \left( x_1, x_2, x_3, \dots \right) = \left( 0, x_2, x_3, \dots \right)
\]
which has 0 as an eigenvalue (with eigenspace $\left\{ \left( x, 0, 0, \dots \right) : \, x \in \R \right\}$). \qedhere

\end{enumerate}
\end{proof}

\item Recall that a matrix $M$ representing a linear operator in $L(V)$ is \emph{diagonizable} if there exists a basis of $V$ with respect to which $M$ has nonzero entries only on its diagonal.
Show that the matrix $\begin{bmatrix} a&b\\ 0&d \end{bmatrix}$ is diagonizable if and only if $a \ne d$ or $b=0$.

\begin{proof}
If $a \ne d$ then (since $M$ is upper triangular) there are two linearly independent eigenvectors, which thus form a basis of $V$; writing the linear operator with respect to this basis yields a diagonal matrix.
If $b=0$ then $M$ is diagonal.

Conversely, suppose $a=d$ and $b \ne 0$. Again $M$ is upper triangular, and so by theorem in class, $a$ is the only eigenvalue.
The corresponding eigenspace
\[
  \null (M - a \, \Id) = \null \begin{bmatrix} 0&b\\ 0&0 \end{bmatrix}
\]
is one-dimensional, and thus there does not exist a basis of $V$ consisting of eigenvectors; consequently, $M$ is not diagonalizable.
\end{proof}

\item Let $M$ and $N$ be $n \times n$ matrices, and let $D$ be a diagonal $n \times n$ matrix, all with entries in $\C$. Prove:
  \begin{enumerate}
  \item $MN = ND$ if and only if the diagonal elements of $D$ are eigenvalues of $M$ and the columns of $N$ are the corresponding eigenvectors.
  \item If $N$ is invertible and $M = N D N^{ -1 }$ then $M$ is diagonalizable.
  \end{enumerate}

\begin{proof}
\begin{enumerate}

\item For a matrix $A$, we use the notation $A[k]$ to denote the $k$'th column vector of $A$ and $A[j,k]$ to denote the $(j,k)$-entry of $A$.
Note that $(AB)[k] = A(B[k])$, and if $D$ is diagonal, $(AD)[k] = D[k,k] \, A[k]$.

If $MN = ND$, then
\[
  M(N[k]) = (MN)[k] = (ND)[k] = D[k,k] \, N[k] \, ,
\]
in other words, $N[k]$ is an eigenvector of $M$ with eigenvalue $D[k,k]$.

Converseley, if the $D[k,k]$'s are the eigenvalues of $M$, each coming with the eigenvector $N[k]$, then
\[
  (MN)[k] = M(N[k]) = D[k,k] \, N[k] = (ND)[k] \, ,
\]
that is, the columns of the matrices $MN$ and $ND$ are identical. But that means $MN=ND$.

\item Suppose $N$ is invertible and $M = N D N^{ -1 }$.
The latter implies (by multiplying $N$ on the right) $MN=ND$, and so by (a), the columns of $N$ are the eigenvectors of $M$.
Since $N$ is invertible, these eigenvectors form a basis of $\F^n$, and so (by a theorem proved in class) $M$ is diagonizable. \qedhere

\end{enumerate}
\end{proof}

\item Let $U := \left\{ \left( x_1, x_2, x_3, x_4 \right) \in \R^4 : \, x_1 + x_2 + x_3 + x_4 = 0 = x_1 + 2 x_2 + 3 x_3 + 4 x_4 \right\}$.
Explicitly construct two different projections $P$ and $Q$ from $\R^4$ onto $U$ by giving matrix representations of $P$ and $Q$ with respect to the standard basis.

\begin{proof}
Given any basis $(\u_1, \u_2)$ of the subspace $U$, the matrices
\[
  P = \left[ \u_1 \ \u_2 \ \u_1 \ \u_2 \right]
  \qquad \text{ and } \qquad
  P = \left[ \u_1 \ \u_2 \ \u_2 \ \u_1 \right]
\]
(written in terms of their columns) will be two such projections.
One such basis is formed by $\u_1 = (1,-2,1,0)$ and $\u_2 = (2,-3,0,1)$.

(Here's a generic way to obtain such a basis:
Let $u = \left( x_1, x_2, x_3, x_4 \right) \in U$; then $x_1 + x_2 + x_3 + x_4 = 0 = x_1 + 2 x_2 + 3 x_3 + 4 x_4$.
Thus $x_2 + 2 x_3 + 3 x_4 = 0$, and so we can express $x_1$ and $x_2$ in terms of $x_3$ and $x_4$:
\begin{align*}
  x_2 &= -2 x_3 - 3 x_4 \\
  x_1 &= 2 x_3 + 3 x_4 - x_3 - x_4 = x_3 + 2 x_4 \, .
\end{align*}
So we can write
\[
  U = \left\{ \left( a+2b, -2a-3b, a, b \right) \in \R^4 : \, a, b \in \R \right\}
\]
and our above basis comes from choosing $(a,b)$ to be $(1,0)$ and $(0,1)$.
\end{proof}

\end{enumerate}

\end{document}