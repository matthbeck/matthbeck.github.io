\documentclass[11pt]{amsart}
\usepackage{amssymb,amsmath,latexsym,enumerate,mathptmx,microtype}
\hoffset=0in 
\voffset=0in
\oddsidemargin=0in
\evensidemargin=0in
\topmargin=-.2in 
\textwidth=6.5in
\textheight=9in
\begin{document}
\setlength{\parindent}{0pt}
\setlength{\parskip}{0.4cm}
\thispagestyle{empty} 
\def\Id{\mathrm I}
\def\0{\mathbf 0}
\def\b{\mathbf b}
\def\c{\mathbf c}
\def\e{\mathbf e}
\def\f{\mathbf f}
\def\u{\mathbf u}
\def\v{\mathbf v}
\def\w{\mathbf w}
\def\x{\mathbf x}
\def\y{\mathbf y}
\def\C{\mathbf{C}}
\def\F{\mathbf{F}}
\def\R{\mathbf{R}}
\def\Z{\mathbf{Z}}
\def\P{\mathcal{P}}
\newcommand\spn{\operatorname{span}}
\renewcommand\null{\operatorname{null}}
\newcommand\range{\operatorname{range}}
\newcommand\rank{\operatorname{rank}}
\newcommand\mult{\operatorname{mult}}
\newcommand\norm[1]{\left|\left| #1 \right|\right|}
\newcommand\inner[2]{\left< #1, #2 \right>}

\begin{center} {\bf MATH 725 \qquad \qquad Homework Set 12 \qquad \qquad due 11/28/11} \end{center} 

\begin{enumerate}[(1)]

\vspace{12pt}

\item Suppose $V$ is an $n$-dimensional complex vector space and $T \in L(V)$. Prove:
  \begin{enumerate}
  \item If $\null(T^{ n-1 }) \ne \null(T^n)$ then $T$ is nilpotent.
  \item If $\null(T^{ n-2 }) \ne \null(T^{ n-1 })$ then $T$ has at most two distinct eigenvalues.\footnote{\emph{Hint:} think about the multiplicity of the eigenvalue~0.}
  \end{enumerate}

\begin{proof}
\begin{enumerate}

\item We know from class that
\[
  \{ \0 \} = \null(T^0) \subseteq \null(T) \subseteq \null(T^2) \subseteq \cdots
\]
and this this series starts with strict inclusions and it stabilizes with the first equality.
So if $\null(T^{ n-1 }) \ne \null(T^n)$ then all set inclusions up to $\null(T^n)$ are strict (and then we have equality); however, this means that the dimension goes up by exactly one with each set inclusion. But this implies $\dim \null(T^n) = n$, and so $\null(T^n) = V$, which means that $T$ is nilpotent.

\item By the same argument as in (a), we conclude from $\null(T^{ n-2 }) \ne \null(T^{ n-1 })$ that $\dim \null(T^{ n-1 }) \ge n-1$.
This means that 0 is an eigenvalue of multiplicity $\ge n-1$, and since the sum of the multiplicities of all eigenvalues of $T$ is $n$, there can be at most one more eigenvalue. \qedhere

\end{enumerate}
\end{proof}

\item Suppose $V$ is an $n$-dimensional complex vector space (where $n \ge 2$), and $T \in L(V)$ has (only) the eigenvalues $\lambda$ and $\mu$. Prove that $(T - \lambda \, \Id)^{ n-1 } (T - \mu \, \Id)^{ n-1 }$ is the zero operator.

\begin{proof}
The characteristic polynomial of $T$ is $c(x) := (x-\lambda)^d (x-\mu)^e$, where $d$ and $e$ are the multiplicities of the eigenvalues $\lambda$ and $\mu$, respectively.
We know that $d, e \ge 1$ and $d+e = n$, which implies that $d, e \le n-1$.
Thus $c(x)$ divides the polynomial $p(x) := (x-\lambda)^{ n-1 } (x-\mu)^{ n-1 }$, and since $c(T) = 0$ (by the Cayley--Hamilton theorem), we conclude $p(T) = 0$.
\end{proof}

\item Suppose $V$ is a complex vector space and $T \in L(V)$. Prove that $V$ has a basis consisting of eigenvectors of $V$ if and only if every generalized eigenvector of $T$ is an eigenvector of~$T$.

\begin{proof}
Suppose every generalized eigenvector of $T$ is an eigenvector of~$T$. We proved in class that we can always find a basis of $V$ consisting of generalized eigenvectors, and so in this case this is a basis of eigenvectors.

Conversely, suppose $V$ has a basis $\left( \v_1, \v_2, \dots, \v_n \right)$ consisting of eigenvectors of $T$, and let $\lambda_j$ be the eigenvalue corresponding to $\v_j$. Now, let $\v$ be a generalized eigenvector for the eigenvalue $\lambda$. Since $\left( \v_1, \v_2, \dots, \v_n \right)$ is a basis, we can write
$
  \v = \sum_{ j=1 }^n a_j \, \v_j
$
for some $a_1, a_2, \dots, a_n \in \C$. Since $\v$ is a generalized eigenvector,
\[
  \0
  = (T - \lambda \, \Id)^n (\v)
  = \sum_{ j=1 }^n a_j (T - \lambda \, \Id)^n (\v_j)
  = \sum_{ j=1 }^n a_j (\lambda_j - \lambda)^n \, \v_j \, ,
\]
and so (because $\left( \v_1, \v_2, \dots, \v_n \right)$ is linear independent) $a_j (\lambda_j - \lambda)^n = 0$ for all $j$.
Thus $a_j = 0$ for all $j$ such that $\lambda_j \ne \lambda$, and so $\v$ is a linear combination of eigenvectors corresponding to the eigenvalue $\lambda$. But this implies that $\v$ is also an eigenvector.
\end{proof}

\item Prove that if $T \in L(V)$ is invertible then there exists a polynomial $p(x) \in \P(\F)$ such that $T^{ -1 } = p(T)$.\footnote{\emph{Hint:} start by proving that the minimal polynomial of $T$ has a nonzero constant term.}

\begin{proof}
Let $m(x) = a_d x^d + a_{ d-1 } x^{ d-1 } + \dots + a_0$ be the minimal polynomial of $T$.
If $a_0 = 0$, the polynomial $\frac{ m(x) }{ x }$ (which has degree $d-1$) applied to $T$ would give the zero operator, contradicting the minimality of $d$. Thus $a_0 \ne 0$.

By the Cayley--Hamilton theorem, $m(T) = 0$, which implies
\[
  a_d T^d + a_{ d-1 } T^{ d-1 } + \dots + a_1 T = - a_0 \, \Id \, ,
\]
which in turn we can write as
\[
  T \left( - \frac{ a_d }{ a_0 } T^{ d-1 } - \frac{ a_d }{ a_0 } T^{ d-2 } - \dots - \frac{ a_1 }{ a_0 } \right) = \Id \, .
\]
Thus $T^{ -1 } = p(T)$ where
\[
  p(x) = - \frac{ a_d }{ a_0 } x^{ d-1 } - \frac{ a_d }{ a_0 } x^{ d-2 } - \dots - \frac{ a_1 }{ a_0 } \, . \qedhere
\]
\end{proof}

\item Give an example of an operator $T \in L(V)$ whose minimal and characteristic polynomials both equal $x (x-29)^2 (x-34)$.

\begin{proof}[Solution]
One example is $T \in L(\C^4)$ given by the matrix (with respect to the standard basis of $\C^4$)
\[
  \left( \begin{array}{cccc}
  0 & 0 & 0 & 0 \\
  0 & 29 & 1 & 0 \\
  0 & 0 & 29 & 0 \\
  0 & 0 & 0 & 34
  \end{array} \right) .
\]
Then 0 is an eigenvalue (with eigenspace spanned by the first basis vector)
and 34 is an eigenvalue (with eigenspace spanned by the last basis vector), both with multiplicity 1.
The last eigenvalue is 29 (visible from the diagonal form), with eigenspace spanned by the second basis vector. It's multiplicity is 2 because, while $\dim \null(T - 29 \, \Id) = 1$, the null space of
\[
  (T - 29 \, \Id)^2 =
  \left( \begin{array}{cccc}
  841 & 0 & 0 & 0 \\
  0 & 0 & 0 & 0 \\
  0 & 0 & 0 & 0 \\
  0 & 0 & 0 & 25
  \end{array} \right)
\]
has dimension 2, as do $\null(T - 29 \, \Id)^j$ for $j \ge 2$.
Thus the characteristic polynomial of $T$ is $x (x-29)^2 (x-34)$, and since the minimal polynomial divides the characteristic polynomial, it can either be $x (x-29) (x-34)$ or $x (x-29)^2 (x-34)$. One easily checks that $T (T-29) (T-34)$ applied to the third basis vector is not zero, and so $T (T-29) (T-34)$ cannot be the zero operator. Thus the minimal polynomial of $T$ is also $x (x-29)^2 (x-34)$.
\end{proof}

\end{enumerate}

\end{document}