\documentclass[11pt]{amsart}
\usepackage{amssymb,amsmath,latexsym,enumerate,mathptmx,microtype}
\hoffset=0in 
\voffset=0in
\oddsidemargin=0in
\evensidemargin=0in
\topmargin=-.2in 
\textwidth=6.5in
\textheight=9in
\begin{document}
\setlength{\parindent}{0pt}
\setlength{\parskip}{0.4cm}
\thispagestyle{empty} 
\def\Id{\mathrm I}
\def\0{\mathbf 0}
\def\b{\mathbf b}
\def\c{\mathbf c}
\def\e{\mathbf e}
\def\u{\mathbf u}
\def\v{\mathbf v}
\def\w{\mathbf w}
\def\x{\mathbf x}
\def\y{\mathbf y}
\def\C{\mathbf{C}}
\def\F{\mathbf{F}}
\def\R{\mathbf{R}}
\def\Z{\mathbf{Z}}
\def\P{\mathcal{P}}
\newcommand\spn{\operatorname{span}}
\renewcommand\null{\operatorname{null}}
\newcommand\range{\operatorname{range}}
\newcommand\rank{\operatorname{rank}}
\newcommand\norm[1]{\left|\left| #1 \right|\right|}
\newcommand\inner[2]{\left< #1, #2 \right>}

\begin{center} {\bf MATH 725 \qquad \qquad Homework Set 8 \qquad \qquad due 10/17/11} \end{center} 

\begin{enumerate}[(1)]

\item Let $U = \spn \left( (1,1,1,0), (1,1,0,2) \right) \subseteq \R^4$.
Find the vector $\u \in U$ that minimizes $\norm{ \u - (1,2,3,4) }$.

\begin{proof}[Solution]
We consider $\R^4$ with the usual inner product and find an orthnormal basis of $U$: use Gram--Schmidt on $\left( (1,1,1,0), (1,1,0,2) \right)$ to obtain the orthonormal basis $(\e_1, \e_2)$ where
\[
  \e_1 := \tfrac{ 1 }{ \sqrt 3 } (1,1,1,0)
\]
and
\[
  \e_2
  := \frac{ (1,1,0,2) - \inner{ (1,1,0,2) }{ \e_1 } \e_1 }{ \norm{ (1,1,0,2) - \inner{ (1,1,0,2) }{ \e_1 } \e_1 } }
  = \sqrt{ \tfrac{ 3 }{ 14 } } \left( \tfrac 1 3, \tfrac 1 3, - \tfrac 2 3, 2 \right) .
\]
We proved in class that the projection of $\v \in \R^4$ onto $U$ is then given by
\[
  \u := \inner \v {\e_1} \e_1 + \inner \v {\e_2} \e_2
\]
and this $\u$ minimizes the distance to $\v$ amond all vectors in $U$.
For $\v = (1,2,3,4)$, this is
\begin{align*}
  \u
  &= \inner{ (1,2,3,4) }{ \tfrac{ 1 }{ \sqrt 3 } (1,1,1,0) } \tfrac{ 1 }{ \sqrt 3 } (1,1,1,0) + \inner{ (1,2,3,4) }{ \sqrt{ \tfrac{ 3 }{ 14 } } \left( \tfrac 1 3, \tfrac 1 3, - \tfrac 2 3, 2 \right) } \sqrt{ \tfrac{ 3 }{ 14 } } \left( \tfrac 1 3, \tfrac 1 3, - \tfrac 2 3, 2 \right) \\
  &= \left( \tfrac 5 2, \tfrac 5 2, 1, 3 \right) \qedhere
\end{align*}

\end{proof}

\item Find the polynomial $p(x) \in \P_3(\R)$ with zero constant and linear coefficient that minimizes
\[
  \int_{ -1 }^1 \left| 2+3x-p(x) \right|^2 dx \, .
\]

\begin{proof}[Solution]
We use our inner product from last week:
\[
  \inner f g := \int_{ -1 }^1 f(x) \, g(x) \, dx \, .
\]
Let $U := \left\{ a \, x^3 + b \, x^2 : \, a, b \in \R \right\}$ and $q(x) := 2+3x$; then the problem asks for the closest point $p \in U$ to $q$, i.e., the orthogonal projection of $q$ onto $U$.

We first compute an orthonormal basis of $U$.
A basis of $U$ is $\left( x^2, x^3 \right)$; applying Gram--Schmidt this basis gives
\[
  e_1(x) = \frac{ x^2 }{ \norm{ x^2 } } = \sqrt{ \tfrac 5 2 } \, x^2
\]
and
\[
  e_2(x)
  = \frac{ x^3 - \inner{ x^3 }{ e_1(x) } e_1(x) }{ \norm{ x^3 - \inner{ x^3 }{ e_1(x) } e_1(x) } }
  = \frac{ x^3 }{ \norm{ x^3 } }
  = \sqrt{ \tfrac 7 2 } \, x^3 \, .
\]
We know that the projection of $q$ onto $U$ is
\[
  p = \inner{ q }{ e_1 } e_1 + \inner{ q }{ e_2 } e_2
  = \tfrac 5 2 \, x^2 \int_{ -1 }^1 (2+3x) \, x^2 \, dx + \tfrac 7 2 \, x^3 \int_{ -1 }^1 (2+3x) \, x^3 \, dx
  = \tfrac{ 10 }{ 3 } \, x^2 + \tfrac{ 21 }{ 5 } \, x^3 . \qedhere
\]
\end{proof}

\item Suppose $T \in L(V)$. Prove that if $T^2 = T$ and every vector in $\null(T)$ is orthogonal to every vector in $\range(T)$, then $T$ is an orthogonal projection onto $\range(T)$.

\begin{proof}
Suppose that $T^2 = T$ and every vector in $\null(T)$ is orthogonal to every vector in $\range(T)$.
Then we can write any $\v \in V$ as
\[
  \v = T(\v) + \left( \v - T(\v) \right) .
\]
The first summand on the right-hand side is in $\range(T)$, and so we're done if we can prove that the second summand, $\v - T(\v)$, is in $\range(T)^\perp$. But this follows from
\[
  T(\v - T(\v)) = T(\v) - T^2(\v) = 0 \, ,
\]
which means that $\v - T(\v)$ is in $\null(T)$ and thus orthogonal to every vector in $\range(T)$.
\end{proof}

\item Suppose $T \in L(V)$.
  \begin{enumerate}
  \item Prove that $\lambda$ is an eigenvalue of $T$ if and only if $\overline \lambda$ is an eigenvalue of $T^*$.
  \item Show that $U$ is invariant under $T$ if and only if $U^\perp$ is invariant under $T^*$.
  \end{enumerate}

\begin{proof}
\begin{enumerate}

\item We will prove the contrapositive (in both directions).
$\lambda$ is not an eigenvalue of $T$ if and only if $T - \lambda \Id$ is invertible, which means there exists $S \in L(V)$ such that
\[
  S(T - \lambda \Id) = (T - \lambda \Id)S = \Id \, .
\]
This, in turn, is equivalent to
\[
  (T - \lambda \Id)^* S^* = S^* (T - \lambda \Id)^* = \Id \, ,
\]
which means that $(T - \lambda \Id)^* = T^* - \overline \lambda \Id$ is invertible, which holds if and only if $\overline \lambda$ is not an eigenvalue of $T^*$.

\item We only have to prove one implication, since $(U^\perp)^\perp = U$ and $(T^*)^* = T$.
Suppose $U$ is invariant under $T$. Given $\v \in U^\perp$, we need to show that $T^*(\v) \in U^\perp$, i.e., that $\inner \u {T^*(\v)} = 0$ for all $\u \in U$. But for any such $\u \in U$,
\[
  \inner \u {T^*(\v)} = \inner {T(\u)} \v = 0
\]
because $T(\u) \in U$ and $\v \in U^\perp$. \qedhere

\end{enumerate}
\end{proof}

\item Suppose $A \in L(\R^n, \R^m)$ is given by a matrix (written in terms of the standard bases) whose columns are linearly independent, and let $\b \in \R^m$.
If $A$ is not surjective, the linear system $A \x = \b$ might not have a solution $\x \in \R^n$.
The following exercise computes the best approximation (in the sense of part (c)) to a solution.
  \begin{enumerate}
  \item Show that $A$ is injective.
  \item Show that $A^* A$ is invertible.
  \item Prove that $\y := (A^* A)^{-1} A^* \, \b$ satisfies $\norm{ A \y - \b } \le \norm{ A \x - \b }$ for all $\x \in \R^n$.\footnote{
\emph{Hint:} Start by proving that $A (A^* A)^{ -1 } A^*$ is the orthogonal projection onto $\range(A)$.}
  \end{enumerate}

\begin{proof}
\begin{enumerate}

\item Suppose the columns of $A$ are $\c_1, \c_2, \dots, \c_n \in \R^m$.
A vector $\left( v_1, v_2, \dots, v_n \right) \in \null(A)$ satisfies
\[
  v_1 \, \c_1 + v_2 \, \c_2 + \dots v_n \, \c_n = \0 \, ,
\]
and so $v_1 = v_2 = \dots = v_n = 0$ because $\c_1, \c_2, \dots, \c_n$ are linearly independent.
Thus $\null(A) = \{\0\}$.

\item Since $A^* A \in L(\R^n)$, so it suffices to prove that $\null(A^* A) = \{\0\}$.
Suppose $A^* A \v = \0$. Then
\[
  0 = \inner \v { A^* A \v } = \inner{ A \v }{ A \v }
\]
and so $A(\v) = \0$. Part (a) implies that $\v = \0$.
Thus $\null(A^* A) = \{\0\}$.

\item We claim that $P := A (A^* A)^{ -1 } A^*$ is the orthogonal projection onto $\range(A)$.

Note that $A^*$ is surjective (because $A$ is injective), as is $(A^* A)^{ -1 }$ (because it is invertible), and so $\range(P) = \range(A)$.
Thus, by Exercise (3), it suffices to prove that $P^2 = P$ and that every vector in $\null(P)$ is orthogonal to every vector in $\range(A)$.

The former assertion follows with $P^2 = A \left( (A^* A)^{ -1 } A^* A \right) (A^* A)^{ -1 } A^* = A (A^* A)^{ -1 } A^* = P$.

To prove that every vector in $\null(P)$ is orthogonal to every vector in $\range(A)$, suppose $\v \in \null(P)$.
Then $A (A^* A)^{ -1 } A^* \, \v = \0$, i.e., $(A^* A)^{ -1 } A^* \, \v \in \null(A)$, which implies by (a) that $(A^* A)^{ -1 } A^* \, \v = \0$, which implies by (b) that $A^* \, \v = \0$, i.e., $\v \in \null(A^*)$. We proved in class that $\null(A^*) = \range(A)^\perp$, and so $\v$ is orthogonal to every vector in $\range(A)$.

Thus we have proved that $P$ is the orthogonal projection onto $\range(A)$.
But this means, given $\b \in \R^m$, $P \, \b$ is closest (measured using the norm) to $\b$ among all the vectors in $\range(A)$, in other words,
\[
  \norm{ A \y - \b } = \norm{ P \, \b - \b } \le \norm{ A \x - \b }
\]
for all $\x \in \R^n$. \qedhere

\end{enumerate}
\end{proof}

\end{enumerate}

\end{document}