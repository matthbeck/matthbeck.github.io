\documentclass[11pt]{amsart}
\usepackage{amssymb,amsmath,latexsym,enumerate,mathptmx,microtype}
\hoffset=0in 
\voffset=0in
\oddsidemargin=0in
\evensidemargin=0in
\topmargin=-.7in 
\textwidth=6.5in
\textheight=9.5in
\begin{document}
\setlength{\parindent}{0pt}
\setlength{\parskip}{0.4cm}
\thispagestyle{empty} 
\def\s{\mathbf{s}}
\def\t{\mathbf{t}}
\def\u{\mathbf{u}}
\def\v{\mathbf{v}}
\def\C{\mathbf{C}}
\def\F{\mathbf{F}}
\def\R{\mathbf{R}}
\def\Z{\mathbf{Z}}
\def\P{\mathcal{P}}
\newcommand\spn{\operatorname{span}}

\begin{center} {\bf MATH 725 \qquad \qquad Homework Set 2 \qquad \qquad due 9/7/11} \end{center} 

\begin{enumerate}[(1)]

\item Consider the subspaces
\begin{align*}
  U &:= \left\{ (x,y,z,w) \in \R^4 : \, x+y+z+w=x-y+z-w=0 \right\} \qquad \text{ and } \\
  V &:= \left\{ (x,y,z,w) \in \R^4 : \, x+z=y-w=0 \right\}
\end{align*}
of $\R^4$.
Is $\R^4 = U \oplus V$? If so, prove it; if not, construct a subspace $W$ such that $\R^4 = U \oplus W$.

\begin{proof}[Solution]
First note that the equations $x+y+z+w=x-y+z-w=0$ are equivalent to the equations $x+y+z+w=0$ and $y+w=-y-w$, the latter of which is equivalent to $y+w=0$, which in turn implies that the former equation can be simplified to $x+z = 0$. So
\[
  U = \left\{ (x,y,z,w) \in \R^4 : \, x+z = y+w = 0 \right\} .
\]
The vector $(1,0,-1,0)$ is in both $U$ and $V$, and so $U+V$ is not a direct product (because then we would have $U \cap V = \left\{ (0,0,0,0) \right\}$).

Let $W := \left\{ (x,y,z,w) \in \R^4 : \, z = w = 0 \right\}$. We claim that $\R^4 = U \oplus W$; we will prove this by showing $\R^4 = U + W$ and $U \cap W = \left\{ (0,0,0,0) \right\}$.
First, any $(a,b,c,d) \in \R^4$ can be written as
\[
  (a,b,c,d) = (-c,-d,c,d) + (a+c,b+d,0,0) \, ;
\]
note that $(-c,-d,c,d) \in U$ and $(a+c,b+d,0,0) \in W$, so this proves $\R^4 = U + W$.
Now suppose $(a,b,c,d) \in U \cap W$; then
\[
  a = -c = 0
  \qquad \text{ and } \qquad
  b = -d = 0 \, ,
\]
that is, $(a,b,c,d) = (0,0,0,0)$. This proves $U \cap W = \left\{ (0,0,0,0) \right\}$.
\end{proof}

\item Let $U := \left\{ \left( x_1, x_2, \dots, x_5 \right) \in \R^5 : \, x_1 = 5 x_2 = 6 x_3 \right\}$.
  \begin{enumerate}
  \item Construct a basis of $U$ (and prove that it is a basis).
  \item Construct a basis of $\R^5$ that extends your basis in (a).
  \end{enumerate}

\begin{proof}[Solution]
\begin{enumerate}

\item We claim that $B := \left( (30,6,5,0,0), (0,0,0,1,0), (0,0,0,0,1) \right)$ is a basis for $U$.
The list $B$ is linearly independent because
\[
  a (30,6,5,0,0) + b (0,0,0,1,0) + c (0,0,0,0,1) = (30a, 6a, 5a, b, c) = (0,0,0,0,0)
\]
implies that $a = b = c = 0$.
The list $B$ spans $U$ because any vector in $U$ is by definition of the form $\left( x_1, \tfrac 1 5 x_1 , \tfrac 1 6 x_1, x_4, x_5 \right)$, and
\[
  \left( x_1, \tfrac 1 5 x_1 , \tfrac 1 6 x_1, x_4, x_5 \right) = x_1 (30,6,5,0,0) + x_4 (0,0,0,1,0) + x_5 (0,0,0,0,1) \, .
\]

\item We claim that $C := \left( (30,6,5,0,0), (0,0,0,1,0), (0,0,0,0,1), (1,0,0,0,0), (0,1,0,0,0) \right)$ is such a basis for $\R^5$.
The list $C$ is linearly independent because
\begin{align*}
  &a (30,6,5,0,0) + b (0,0,0,1,0) + c (0,0,0,0,1) + d (1,0,0,0,0) + e (0,1,0,0,0) \\
  &\qquad = (30a+d, 6a+e, 5a, b, c) = (0,0,0,0,0)
\end{align*}
implies that $a = b = c = d = e = 0$.
The list $C$ spans $\R^5$ because
\begin{align*}
  \left( x_1, x_2, x_3, x_4, x_5 \right) = &\tfrac 1 5 x_3 (30,6,5,0,0) + x_4 (0,0,0,1,0) + x_5 (0,0,0,0,1) + \\
  &\qquad + \left( x_1 - 6 x_3 \right) (1,0,0,0,0) + \left( x_2 - \tfrac 6 5 x_3 \right) (0,1,0,0,0) \, . \qedhere
\end{align*}
\end{enumerate}
\end{proof}

\item Suppose $S_1,S_2 \subseteq V$. Let $U_1 := \spn(S_1)$ and $U_2 := \spn(S_2)$.
  \begin{enumerate}
  \item Show that $U_1 = U_2$ if and only if $S_1 \subseteq U_2$ and $S_2 \subseteq U_1$.
  \item Show that $\spn \left( S_1 \cup S_2 \right) = U_1 + U_2$.
  \end{enumerate}

\begin{proof}[Solution]
\begin{enumerate}

\item Assume $U_1 = U_2$.
Any $\s \in S_1$ certainly lies in $\spn(S_1) = U_2$, and so $S_1 \subseteq U_2$.
Switching the subscripts yields $S_2 \subseteq U_1$.

Conversely, assume $S_1 \subseteq U_2$ and $S_2 \subseteq U_1$.
Any $\u \in U_1$ can be written as a linear combination of vectors in $S_1$, and since $S_1 \subseteq U_2$, this linear combination is in $U_2$, i.e., $\u \in U_2$. This gives $U_1 \subseteq U_2$.
Switching the subscripts yields $U_2 \subseteq U_1$.

\item Since $S_1 \subseteq U_1$ and $S_2 \subseteq U_2$, any vector in $\spn \left( S_1 \cup S_2 \right)$ is in $U_1 + U_2$, i.e., $\spn \left( S_1 \cup S_2 \right) \subseteq U_1 + U_2$.

Now let $\u = \u_1 + \u_2 \in U_1 + U_2$. Then we can write $\u_1$ and $\u_2$ as linear combinations of vectors in $S_1 = \left( \s_1, \s_2, \dots, \s_m \right) $ and $S_2 = \left( \t_1, \t_2, \dots, \t_n \right) $, respectively, say
\[
  \u_1 = \sum_{ j=1 }^m a_j \, \s_j
  \qquad \text{ and } \qquad
  \u_2 = \sum_{ j=1 }^n b_j \, \t_j \, .
\]
The lists $S_1$ and $S_2$ might have vectors in common; upon possibly renumbering the two lists, we may assume that the common vectors are $\s_1 = \t_1, \s_2 = \t_2, \dots, \s_k = \t_k$. Thus we can rewrite
\[
  \u_1 = \sum_{ j=1 }^k a_j \, \s_j + \sum_{ j=k+1 }^m a_j \, \s_j
  \qquad \text{ and } \qquad
  \u_2 = \sum_{ j=1 }^k b_j \, \s_j + \sum_{ j=k+1 }^n b_j \, \t_j
\]
and so
\[
  \u = \u_1 + \u_2 = \sum_{ j=1 }^k \left( a_j + b_j \right) \s_j + \sum_{ j=k+1 }^m a_j \, \s_j + \sum_{ j=k+1 }^n b_j \, \t_j
\]
is in $\spn \left( S_1 \cup S_2 \right) = \spn \left( \s_1, \s_2, \dots, \s_m, \t_{ k+1 } , \t_{ k+2 } , \dots, \t_n \right)$. This proves $U_1 + U_2 \subseteq \spn \left( S_1 \cup S_2 \right)$. \qedhere

\end{enumerate}
\end{proof}

\item Recall the definition of the binomial coefficient
\[
  \binom x n := \frac{ x (x-1) (x-2) \cdots (x-n+1) }{ n! }
\]
for an arbitrary $x$ (e.g., $x \in \C$ or $x$ a variable) and $n \in \Z_{ \ge 0 }$.
Show that $\left( \binom x 0 , \binom x 1 , \dots, \binom x n \right)$ is a basis of $\P_n(\F)$, the set of all polynomials of degree $\le n$ with coefficients in~$\F$.

\begin{proof}
Let $S_1 := \left( 1, x, x^2, \dots, x^n \right)$ and $S_2 := \left( \binom x 0 , \binom x 1 , \dots, \binom x n \right)$.
The list $S_1$ is a basis of $\P_n(\F)$ by inspection, so by Exercise (3a), we only need to show that (viewed as sets) $S_1 \subseteq \spn(S_2)$ and $S_2 \subseteq \spn(S_1)$.
The latter set inclusion is clear by expanding $\binom x j$ in terms of $x$.

To prove $S_1 \subseteq \spn(S_2)$, we need to show that $x^k \in \spn(S_2)$ for any $0 \le k \le n$.
We will prove this by induction on $n$.
The base case is $n=0$: $1 \in \spn \left\{ \binom x 0 \right\} = \spn \left\{ 1 \right\}$.
For the induction step, assume that $x^j \in \spn \left( \binom x 0 , \binom x 1 , \dots, \binom x {n-1} \right)$ for any $0 \le j \le n-1$.
Now given $0 \le k \le n$, if $k \ne n$ then $x^k \in \spn \left( \binom x 0 , \binom x 1 , \dots, \binom x n \right)$ by induction hypothesis.
If $k = n$ then we use the fact that
\[
  \binom x n = \frac{ 1 }{ n! } \, x^n + p(x)
\]
for some polynomial of degree $n-1$. Thus
\[
  x^n = n! \binom x n + n! \, p(x)
\]
and the last summand can be written as a linear combination of $\binom x 0 , \binom x 1 , \dots, \binom x {n-1}$, by induction hypothesis.
This gives a linear combination for $x^n$ in terms of $\binom x 0 , \binom x 1 , \dots, \binom x n$.
\end{proof}

\item Suppose that $U$ is a subspace of the finite-dimensional vector space $V$, and $\dim U = \dim V$. Prove that $U = V$.

\begin{proof}
Let $n = \dim U = \dim V$.
Since $V$ (and hence also $U$) are finite-dimensional, there exists a basis $B := \left( \v_1, \v_2, \dots, \v_n \right)$ of $U$ that can be extended to a basis of $V$. But since (by a theorem proved in class) such an extended basis also has to contain $n$ vectors, $B$ must already span $V$, that is, $U = V$.
\end{proof}

\end{enumerate}

\end{document}