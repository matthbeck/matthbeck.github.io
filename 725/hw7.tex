\documentclass[11pt]{amsart}
\usepackage{amssymb,amsmath,latexsym,enumerate,mathptmx,microtype}
\hoffset=0in 
\voffset=0in
\oddsidemargin=0in
\evensidemargin=0in
\topmargin=-.2in 
\textwidth=6.5in
\textheight=9in
\begin{document}
\setlength{\parindent}{0pt}
\setlength{\parskip}{0.4cm}
\thispagestyle{empty} 
\def\Id{\mathrm I}
\def\0{\mathbf 0}
\def\e{\mathbf e}
\def\r{\mathbf r}
\def\u{\mathbf u}
\def\v{\mathbf v}
\def\w{\mathbf w}
\def\C{\mathbf{C}}
\def\F{\mathbf{F}}
\def\R{\mathbf{R}}
\def\Z{\mathbf{Z}}
\def\P{\mathcal{P}}
\renewcommand\Re{\operatorname{Re}}
\newcommand\spn{\operatorname{span}}
\renewcommand\null{\operatorname{null}}
\newcommand\range{\operatorname{range}}
\newcommand\rank{\operatorname{rank}}
\newcommand\norm[1]{\left|\left| #1 \right|\right|}
\newcommand\inner[2]{\left< #1, #2 \right>}

\begin{center} {\bf MATH 725 \qquad \qquad Homework Set 7 \qquad \qquad due 10/10/11} \end{center} 

\begin{enumerate}[(1)]

\item Consider the vector space $\P(\C)$ with the inner product $\inner f g := \int_{ -1 }^1 f(x) \overline{g(x)} \, dx \, .$
  \begin{enumerate}
  \item Show that this defines an inner-product space.
  \item Compute the norm of $x^n$, where $n$ is a nonnegative integer.
  \item Compute an orthonormal basis for $\P_2(\C)$.
  \end{enumerate}

\begin{proof}
\begin{enumerate}

\item First, $\inner f f = \int_{ -1 }^1 f(x) \overline{f(x)} \, dx = \int_{ -1 }^1 \left| f(x) \right|^2 dx$, and so this real integral over a nonnegative function is $\ge 0$ and equals 0 if and only if the integrand is the zero function (which is equivalent to $f$ being the zero function).

Second, $\inner{ a f_1 + f_2 } g = \int_{ -1 }^1 \left( a \, f_1(x) + f_2(x) \right) \overline{ g(x) } \, dx = a \int_{ -1 }^1 f_1 \overline{ g(x) } \, dx + \int_{ -1 }^1 f_2 \overline{ g(x) } \, dx$.

Third, $\inner f g = \int_{ -1 }^1 f(x) \overline{g(x)} \, dx = \int_{ -1 }^1 \overline{ \overline{f(x)} g(x) } \, dx = \overline{ \int_{ -1 }^1 \overline{f(x)} g(x) \, dx } = \overline{ \inner g f }$.

\item
\[
  \norm{ x^n }
  = \sqrt{ \inner{ x^n }{ x^n } }
  = \sqrt{ \int_{ -1 }^1 \left| x^n \right|^2 dx }
  = \sqrt{ \int_{ -1 }^1 |x|^{ 2n } dx }
  = \sqrt{ 2 \int_{ 0 }^1 x^{ 2n } dx }
  = \sqrt{ \frac{ 2 }{ 2n+1 } } \, .
\]

\item We apply Gram--Schmidt to the basis $\left( 1, x, x^2 \right)$ of $\P_2(\C)$.
From part (b) we know $\norm 1 = \sqrt 2$, so the first basis vector (polynomial) is $\e_1 = \frac{ 1 }{ \sqrt 2 }$.
To compute $\e_2$, we calculate
\[
  x - \inner x {\e_1} \e_1
  = x - \frac{ 1 }{ 2 } \int_{ -1 }^1 x \, dx
  = x
\]
and so (using part (a)) $\e_2 = \frac{ x }{ \norm x } = \sqrt{ \frac 3 2 } \, x$.
To compute $\e_3$,
\[
  x^2 - \inner {x^2} {\e_1} \e_1 - \inner {x^2} {\e_2} \e_2
  = x^2 - \frac{ 1 }{ 2 } \int_{ -1 }^1 x^2 \, dx - \frac 3 2 \, x \int_{ -1 }^1 x^3 \, dx
  = x^2 - \frac{ 1 }{ 3 }
\]
and so $\e_3 = \frac{ x^2 - \frac{ 1 }{ 3 } }{ \norm{ x^2 - \frac{ 1 }{ 3 } } } = \sqrt{ \frac{ 45 }{ 8 }  }  \left( x^2 - \frac{ 1 }{ 3 } \right) $.

\emph{Remark:} We have just computed the first three \emph{Legendre polynomials}. \qedhere

\end{enumerate}
\end{proof}

\item Suppose that $V$ is an inner-product space over $\F$, and $\v, \w \in V$.
  \begin{enumerate}
  \item If $\F = \R$, show that $\inner \v \w = \frac 1 4 \left( \norm{ \v + \w }^2 - \norm{ \v - \w }^2 \right)$.
  \item If $\F = \C$, show that $\inner \v \w = \frac 1 4 \left( \norm{ \v + \w }^2 - \norm{ \v - \w }^2 + i \norm{ \v + i \, \w }^2 - i \norm{ \v - i \, \w }^2 \right)$.
  \end{enumerate}

\begin{proof}
\begin{enumerate}

\item Computing
\begin{align*}
  \norm{ \v + \w }^2 &= \inner{ \v+\w }{ \v+\w } = \inner \v \v + \inner \v \w + \inner \w \v + \inner \w \w \\
  \norm{ \v - \w }^2 &= \inner{ \v-\w }{ \v-\w } = \inner \v \v - \inner \v \w - \inner \w \v + \inner \w \w
\end{align*}
yields
\[
  \norm{ \v + \w }^2 - \norm{ \v - \w }^2 
  = 2 \inner \v \w + 2 \inner \w \v 
  = 4 \inner \v \w \, ,
\]
where the last equation follows from $\inner \w \v = \overline{ \inner \v \w } = \inner \v \w$ since everything is in~$\R$.

\item Combining the first two equations in (a) with
\begin{align*}
  \norm{ \v + i \, \w }^2 &= \inner{ \v + i \, \w }{ \v + i \, \w } = \inner \v \v - i \inner \v \w + i \inner \w \v + \inner \w \w \\
  \norm{ \v - i \, \w }^2 &= \inner{ \v - i \, \w }{ \v - i \, \w } = \inner \v \v + i \inner \v \w - i \inner \w \v + \inner \w \w
\end{align*}
yields
\[
  \norm{ \v + \w }^2 - \norm{ \v - \w }^2 + i \norm{ \v + i \, \w }^2 - i \norm{ \v - i \, \w }^2
  = 4 \inner \v \w \, . \qedhere
\]

\end{enumerate}
\end{proof}

\item Suppose that $V$ is an inner-product space over $\F$, and $\v, \w \in V$.
Prove that $\inner \v \w = 0$ if and only if $\norm \v \le \norm{\v + a \, \w}$ for all $a \in \F$.

\begin{proof}
Since norms are nonnegative, $\norm \v \le \norm{\v + a \, \w}$ is equivalent to $\norm \v^2 \le \norm{\v + a \, \w}^2$.

Assume $\inner \v \w = 0$, and let $a \in \F$.
Then $\v$ and $a\, \w$ are orthogonal, and so by the Pythagorean theorem,
\[
  \norm{ \v + a \, \w }^2 = \norm \v^2 + \norm{ a \, \w }^2 \ge \norm \v^2 .
\]
Conversely, assume $\norm \v^2 \le \norm{\v + a \, \w}^2$ for all $a \in \F$.
Then
\begin{align*}
  \norm \v^2
  &\le \inner{ \v + a \, \w }{ \v + a \, \w }
  = \inner \v \v + \inner \v {a \, \w} + \inner {a \, \w} \v + \inner{a \, \w}{a \, \w} \\
  &= \norm \v^2 + \overline a \inner \v \w + a \overline{ \inner \v \w } + |a|^2 \norm \w^2
  = \norm \v^2 + 2 \Re \left( \overline a \inner \v \w \right) + |a|^2 \norm \w^2 ,
\end{align*}
i.e.,
\[
  0 \le 2 \Re \left( \overline a \inner \v \w \right) + |a|^2 \norm \w^2 .
\]
If $\w = \0$ then $\inner \v \w = 0$ (as desired), so we may assume that $\w \ne \0$.
Then choosing $a = - \frac{ \inner \v \w }{ \norm \w^2 }$ yields
\begin{align*}
  0
  &\le 2 \Re \left( - \frac{ \overline{ \inner \v \w } }{ \norm \w^2 } \inner \v \w \right) + \left| \frac{ \inner \v \w }{ \norm \w^2 } \right|^2 \norm \w^2
  = 2 \Re \left( - \frac{ \left| \inner \v \w  \right|^2 }{ \norm \w^2 } \right) + \frac{ \left| \inner \v \w  \right|^2 }{ \norm \w^2 } \\
  &= -2 \frac{ \left| \inner \v \w  \right|^2 }{ \norm \w^2 } + \frac{ \left| \inner \v \w  \right|^2 }{ \norm \w^2 }
\end{align*}
which is equivalent to
\[
  2 \left| \inner \v \w  \right|^2 \le \left| \inner \v \w  \right|^2
\]
which in turn can only hold if $\inner \v \w = 0$.
\end{proof}

\item Let $V$ be a real\footnote{The statement of this exercise also holds for \emph{complex} vector spaces, with a similar but more involved proof.} vector space.
A \emph{norm} on $V$ is a function $\norm{\ \ } : V \to \R_{ \ge 0 }$ with the following properties:
  \begin{enumerate}[(i)]
  \item $\norm \v = 0$ if and only if $\v = \0$;
  \item for all $a \in \F$ and all $\v \in V$, $\norm{a \, \v} = |a| \norm \v$;
  \item for all $\v, \w \in V$, $\norm{\v + \w} \le \norm \v + \norm \w$.
  \end{enumerate}
Prove that if a $\norm{\ \ }$ satisfies the ``parallelogram equality"
\[
  \norm{ \v + \w }^2 + \norm{ \v - \w }^2 = 2 \left( \norm{ \v }^2 + \norm{ \w }^2 \right) ,
\]
then it comes from an inner product, i.e., there is an inner product $\inner \ \ $ on $V$ such that for all $\norm \v = \sqrt{ \inner \v \v }$ for all $\v \in V$.

\begin{proof}
Suppose $\norm{\ \ }$ is a norm on $V$ that satisfies the ``parallelogram equality."
Inspired by (2), we define
\[
  \inner \v \w := \tfrac 1 4 \left( \norm{ \v + \w }^2 - \norm{ \v - \w }^2 \right) .
\]
We will prove that this is an inner product on $V$.
First,
\[
  \inner \v \v = \tfrac 1 4 \norm{ 2 \v }^2 = \norm \v
\]
which is $\ge 0$ with equality if and only if $\v = \0$.
Second,
\begin{align*}
  \inner { a \, \v_1 + \v_2 } \w &= \tfrac 1 4 \left( \norm{ a \, \v_1 + \v_2 + \w }^2 - \norm{ a \, \v_1 + \v_2 - \w }^2 \right) \\
  a \inner {\v_1} \w &= \tfrac 1 4 \left( \norm{ a \, \v_1 + \w }^2 - \norm{ a \, \v_1 - \w }^2 \right) \\
  \inner {\v_2} \w &= \tfrac 1 4 \left( \norm{ \v_2 + \w }^2 - \norm{ \v_2 - \w }^2 \right) 
\end{align*}
and so to prove that $\inner { a \, \v_1 + \v_2 } \w = a \inner {\v_1} \w + \inner {\v_2} \w$, we need to show that
\[
  \norm{ a \, \v_1 + \v_2 + \w }^2 - \norm{ a \, \v_1 + \v_2 - \w }^2
  = \norm{ a \, \v_1 + \w }^2 - \norm{ a \, \v_1 - \w }^2
  + \norm{ \v_2 + \w }^2 - \norm{ \v_2 - \w }^2 ,
\]
i.e., that
\[
  \norm{ a \, \v_1 + \v_2 + \w }^2 - \norm{ a \, \v_1 + \v_2 - \w }^2
  + \left( \norm{ a \, \v_1 - \w }^2 + \norm{ \v_2 - \w }^2 \right)
  - \left( \norm{ a \, \v_1 + \w }^2 + \norm{ \v_2 + \w }^2 \right)
  = 0 \, .
\]
To prove this, we will use the ``parallelogram equality" (read from right to left) for the expressions in the two pairs of parantheses, which gives
\begin{align*}
  &\norm{ a \, \v_1 + \v_2 + \w }^2 - \norm{ a \, \v_1 + \v_2 - \w }^2
  + \left( \norm{ a \, \v_1 - \w }^2 + \norm{ \v_2 - \w }^2 \right)
  - \left( \norm{ a \, \v_1 + \w }^2 + \norm{ \v_2 + \w }^2 \right) \\
  &\qquad = \norm{ a \, \v_1 + \v_2 + \w }^2 - \norm{ a \, \v_1 + \v_2 - \w }^2
  + \tfrac 1 2 \left( \norm{ a \, \v_1 + \v_2 - 2 \w }^2 + \norm{ a \, \v_1 - \v_2 }^2 \right) \\
  &\qquad \qquad - \tfrac 1 2 \left( \norm{ a \, \v_1 + \v_2 + 2 \w }^2 + \norm{ a \, \v_1 - \v_2 }^2 \right) \\
  &\qquad = \norm{ a \, \v_1 + \v_2 + \w }^2 - \norm{ a \, \v_1 + \v_2 - \w }^2
  + \tfrac 1 2 \left( \norm{ a \, \v_1 + \v_2 - 2 \w }^2 - \norm{ a \, \v_1 + \v_2 + 2 \w }^2 \right) \\
  &\qquad = \left( \norm{ a \, \v_1 + \v_2 + \w }^2 + \norm \w^2 \right) - \left( \norm{ a \, \v_1 + \v_2 - \w }^2 + \norm \w^2 \right) \\
  &\qquad \qquad + \tfrac 1 2 \left( \norm{ a \, \v_1 + \v_2 - 2 \w }^2 - \norm{ a \, \v_1 + \v_2 + 2 \w }^2 \right) \\
  &\qquad = \tfrac 1 2 \left( \norm{ a \, \v_1 + \v_2 + 2 \w }^2 + \norm{ a \, \v_1 + \v_2 }^2 \right) - \tfrac 1 2 \left( \norm{ a \, \v_1 + \v_2 }^2 + \norm{ a \, \v_1 + \v_2 - 2 \w }^2 \right) \\
  &\qquad \qquad + \tfrac 1 2 \left( \norm{ a \, \v_1 + \v_2 - 2 \w }^2 - \norm{ a \, \v_1 + \v_2 + 2 \w }^2 \right) \\
  &\qquad = 0 \, .
\end{align*}
(In the penultimate equation, we have used the ``parallelogram equality" once more, again read from right to left.)
Third,
\[
  \inner \v \w
  = \tfrac 1 4 \left( \norm{ \v + \w }^2 - \norm{ \v - \w }^2 \right)
  = \tfrac 1 4 \left( \norm{ \w + \v }^2 - \norm{ \w - \v }^2 \right)
  = \inner \w \v \, ,
\]
and this finishes our proof that $\inner \v \w$ is an inner product.
\end{proof}

\item Let $M$ be an real $n \times n$ matrix. Show that the space spanned by the rows of $M$ is the orthogonal complement of $\null(M)$.

\begin{proof}
Suppose the rows of $M$ are $\r_1, \r_2, \dots, \r_n$.
Let $\v \in \spn \left( \r_1, \r_2, \dots, \r_n \right)$, i.e., $\v = \sum_{ j=1 }^{ n } a_j \r_j$ for some $a_1, a_2, \dots, a_n$, and let $\w \in \null(M)$, i.e., $M \w = \0$, i.e., $\inner {\r_j} \w = 0$ for all $j$. Then
\[
  \inner \v \w
  = \inner{ \sum_{ j=1 }^{ n } a_j \r_j } \w
  = \sum_{ j=1 }^{ n } a_j \inner {\r_j} \w = 0 \, . \qedhere
\]
\end{proof}

\end{enumerate}

\end{document}