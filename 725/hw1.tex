\documentclass[11pt]{amsart}
\usepackage{amssymb,amsmath,latexsym,enumerate,mathptmx,microtype}
\hoffset=0in 
\voffset=0in
\oddsidemargin=0in
\evensidemargin=0in
\topmargin=-.7in 
\textwidth=6.5in
\textheight=9.5in
\begin{document}
\setlength{\parindent}{0pt}
\setlength{\parskip}{0.4cm}
\thispagestyle{empty} 
\def\C{\mathbf{C}}
\def\F{\mathbf{F}}
\def\R{\mathbf{R}}
\def\Z{\mathbf{Z}}
\def\P{\mathcal{P}}

\begin{center} {\bf MATH 725 \qquad \qquad Homework Set 1 \qquad \qquad due 8/29/11} \end{center} 

\begin{enumerate}[(1)]

\item Find all solutions (in $\C$) to the following equations: 
  \begin{enumerate} 
  \item $z^6 = 1$
  \item $z^4 = -16$
  \item $z^6=-9$
  \item $z^6-z^3-2=0 \, .$
  \end{enumerate}

\begin{proof}[Solution]
By the fundamental theorem of algebra, a polynomial equation of degree $n$ has exactly $n$ solutions (counting with multiplicities), in which case it thus suffices to give $n$ distinct numbers that satisfy the equation.
  \begin{enumerate} 
  \item $z = e^{ \pi i k/3 }$ for $1 \le k \le 6$ satisfy $z^6 = 1$.
  \item $z = 2 \, e^{ \pi i k/4 }$ for $k = 1, 3, 5, 7$ satisfy $z^4 = -16$.
  \item $z = \sqrt[3] 3 \, e^{ \pi i k/3 }$ for $k = 1, 3, 5$ satisfy $z^6=-9$.
  \item The quadratic equation $x^2 - x - 2 = 0$ has the solutions $x = -1, 2$, and so $z^6-z^3-2=0$ has the solutions $z = e^{ \pi i k/3 }$ for $k = 1, 3, 5$ and $z = \sqrt[3] 2 \, e^{ 2 \pi i k/3 }$ for $k = 1, 2, 3$. \qedhere
  \end{enumerate}
\end{proof}

\item Suppose $U_1, U_2, \dots, U_n$ are subspaces of $V$. Prove that $U_1 + U_2 + \dots + U_n$ is a subspace of $V$.

\begin{proof}
We will check that $0 \in U_1 + U_2 + \dots + U_n$ and that $U_1 + U_2 + \dots + U_n$ is closed under addition and scalar multiplication.

The first assertion follows since $0 \in U_j$ for all $j$, and so $0 = 0 + o + \dots + 0 \in U_1 + U_2 + \dots + U_n$.

Now suppose $u_j, w_j \in U_j$ for each $j$; that is, we have two elements $u_1 + u_2 + \dots + u_n \in U_1 + U_2 + \dots + U_n$ and $w_1 + w_2 + \dots + w_n \in U_1 + U_2 + \dots + U_n$. Since $u_j + w_j \in U_j$ (because $U_j$ is a subspace),
\[
  u_1 + u_2 + \dots u_n + w_1 + w_2 + \dots w_n
  = \left( u_1 + w_1 \right) + \left( u_2 + w_2 \right) + \dots + \left( u_n + w_n \right)
  \in U_1 + U_2 + \dots + U_n \, ,
\]
that is, $U_1 + U_2 + \dots + U_n$ is closed under addition.
Similarly, given $a \in \F$, we know that $a u_j \in U_j$ (again because $U_j$ is a subspace), and so
\[
  a \left( u_1 + u_2 + \dots + u_n \right)
  = a u_1 + a u_2 + \dots + a u_n
  \in U_1 + U_2 + \dots + U_n \, ,
\]
that is, $U_1 + U_2 + \dots + U_n$ is closed under scalar multiplication.
\end{proof}

\item Carefully reason whether or not the following sets are subspaces of $\R^2$:
  \begin{enumerate} 
  \item $\left\{ (a,b) \in \R^2 : \, a, b \ge 0 \right\}$
  \item $\left\{ (a,b) \in \R^2 : \, ab \ge 0 \right\}$
  \item $\left\{ (a,b) \in \R^2 : \, a=b \right\}$
  \item $\Z^2$
  \end{enumerate}

\begin{proof}[Solution]
  \begin{enumerate} 
  \item $(1,1) \in \left\{ (a,b) \in \R^2 : \, a, b \ge 0 \right\}$ but $-(1,1) \notin \left\{ (a,b) \in \R^2 : \, a, b \ge 0 \right\}$, so this is not a subspace.
  \item $(1,1)$ and $(-2,0)$ are both in $\left\{ (a,b) \in \R^2 : \, ab \ge 0 \right\}$ but $(1,1) + (-2,0) = (-1,1)$ is not, so this is not a subspace.
  \item $\left\{ (a,b) \in \R^2 : \, a=b \right\}$ is a subspace: $(0,0)$ is in it, it is closed under addition ($(x,x) + (y,y) = (x+y, x+y)$) and scalar multiplication ($a(x,x) = (ax,ax)$).
  \item $(1,1) \in \Z^2$ but $\frac 1 2 (1,1) \notin \Z^2$, so $\Z^2$ is not a subspace. \qedhere
  \end{enumerate}
\end{proof}

\item Consider the subspace $U := \left\{ p \in \P(\F) : \, \deg(p) \le 3 \right\}$ of the vector space $\P(\F)$ consisting of all polynomials with coefficients in $\F$. Construct a subspace $W$ of $\P(\F)$ such that $\P(\F) = U \oplus W$.

\begin{proof}[Solution]
We claim that $W := \left\{ x^4 p(x) : \, p \in \P(\F) \right\}$ will do the trick (that is, $W$ consists of the zero polynomial and all polynomials that do not have constant, linear, quadratic, or tertiary terms).
$W$ is a subspace because it contains $0$ and is closed under addition and scalar multiplication.
By construction we have $\P(\F) = U+W$ and $U \cap W = \left\{ 0 \right\}$, so by Proposition 1.9, $\P(\F) = U \oplus W$.
\end{proof}

\item Suppose $U$ and $W$ are subspaces of $V$. Prove that $U \cap W$ is the largest subspace of $V$ that is contained in both $U$ and $W$; that is:
  \begin{enumerate} 
  \item $U \cap W$ is a subspace of $V$, and
  \item any other subspace of $V$ that is contained in both $U$ and $W$ is also contained in $U \cap W$.
  \end{enumerate}

\begin{proof}
Since $U$ and $W$ are subspaces, they both contain 0, and so $0 \in U \cap W$.
Given $v_1, v_2 \in U \cap W$, they are both in $U$ and $W$. As subspaces, $U$ and $W$ are closed under addition and scalar multiplication, and so $v_1 + v_2$ is in both $U$ and $V$, that is, $v_1 + v_2 \in U \cap W$; similary, for $a \in \F$, $a v_1$ is in both $U$ and $V$, that is, $a v_1 \in U \cap W$.
Thus $U \cap W$ is also closed under addition and scalar multiplication, and this proves (a).

Now let $S$ be a subspace of $V$ that is contained in both $U$ and $W$. Then (as a set) $S$ is contained in $U \cap W$, and this proves (b).
\end{proof}

\end{enumerate}

\end{document}