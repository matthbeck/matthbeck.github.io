\documentclass[11pt]{article}
\usepackage{amssymb,amsmath,latexsym,enumerate}

\hoffset=0in 
\voffset=0in
\oddsidemargin=0in
\evensidemargin=0in
\topmargin=-.7in 
%\headsep=0in 
%\headheight=0in
\textwidth=6.5in
\textheight=9in

\newcommand\GL{\operatorname{GL}} 
\newcommand\Inn{\operatorname{Inn}} 
\newcommand\Aut{\operatorname{Aut}} 
\newcommand\Stab{\operatorname{Stab}} 
\newcommand\orb{\operatorname{orb}} 
\newcommand\lcm{\operatorname{lcm}} 
\newcommand\gen[1]{\left< #1 \right>}
\def\A{\mathbf{A}}
\def\B{\mathbf{B}}
\def\0{\mathbf{0}}
\def\Z{\mathbb{Z}}
\def\R{\mathbb{R}}
\def\F{\mathbb{F}}
\def\C{\mathbb{C}}
\def\Q{\mathbb{Q}}
\def\m{\mathbf{m}}
\def\x{\mathbf{x}}
\def\y{\mathbf{y}}

\begin{document}
\setlength{\parindent}{0pt}
\setlength{\parskip}{0.4cm}

\thispagestyle{empty} 

\begin{center} {\bf MATH 850 \qquad \qquad Polynomial Method Homework} \end{center} 

\begin{enumerate}[(1)]

\item Fix $d \in \Z_{ >0 }$ and consider $\binom x d$ as a polynomial in $x$ (over your favorite field of characteristic 0).
\begin{enumerate}
  \item Show that, as polynomials, $\binom {-x} d = (-1)^d \binom {x+d-1} d$.
  \item Prove that $\sum_{ n \ge 0 } \binom {n+d} d \, x^n = \frac{ 1 }{ (1-x)^{ d+1 } }$.
  \item Show that $p(x)$ is a polynomial of degree $d$ if and only if
  \[
    \sum_{ n \ge 0 } p(n) \, x^n \ = \ \frac{ h(x) }{ (1-x)^{ d+1 } } 
  \]
  for some polynomial $h(x)$ of degree at most $d$ with $h(1) \ne 0$.
\end{enumerate}

\item A matrix is \emph{unipotent} if it is the sum of the $d \times d$ identity matrix and a \emph{nilpotent}
matrix (i.e., a matrix $\B$ for which there exists a positive integer $k$ such that $\B^k = \0$).
Fix indices $i$ and $j$, and consider the sequence $f(n) := (\A^n)_{ ij }$ formed by the $(i,j)$-entries of the $n$th powers of a unipotent
matrix $\A$. 
Prove that $f(n)$ agrees with a polynomial in~$n$. 
(\emph{Hint:} express $\A^n$ using the binomial theorem.)

\item Let $\Delta$ be the \emph{difference operator} defined by $(\Delta f) (n) := f(n+1) - f(n)$, for a given polynomial $f(n)$.
Prove that $f(n)$ is of degree $\le d$ if and only if $(\Delta^m f)(0) = 0$ for all $m > d$.
(\emph{Hint:} use the \emph{shift operator} $(Sf)(n) := f(n+1)$ and express $(S^n f)(0)$ using the binomial theorem.) 

\item A $d \times d$ matrix with nonnegative integer coefficients is \emph{$n$-magic} if each of its rows and columns sum to $n$.
\begin{enumerate}
  \item Prove that every $n$-magic matrix is the sum of $n$ permutation matrices.
  \item Show that
  \[
    \left\{ (m_{ 11 }, m_{ 12 } , \dots, m_{ dd }, n) \in \Z_{ \ge 0 }^{ d^2+1 } : \, \m \text{ is an $n$-magic matrix} \right\} 
  \]
  forms a semigroup, and deduce that the number $M_d(n)$ of $n$-magic $d \times d$ matrices is a polynomial in~$n$.
\end{enumerate}

\item Compute $M_3(n)$.

\item Let $\m$ be a magic labelling of a given graph $G$ with ``magic sum'' $n$.
\begin{enumerate}
  \item Define the matrix $\A = (a_{ ij })$ where $a_{ ij }$ is the label of $\m$ on the edge connecting the nodes $i$ and $j$. Show that
each row and column sum of $\A$ is $n$, and deduce with the Exercise (4) that
  \[
    2 \A \ = \ \sum_\pi \pi + \pi^T
  \]
  where the sum is over a certain set of permutation matrices.
  \item Prove that $2 \m$ is a sum of magic labellings with magic sum 2, and conclude that every completely fundamental magic labelling of
$G$ has magic sum 1 or~2.
\end{enumerate}

\item Show that $\dim {\rm Poly}_D(\F^n) = \binom{ D+n } n$.

\item We proved in class that, given a finite set $S \subset \R^3$, there is a nonzero polynomial of degree $\le c \, |S|^{ \frac 1 3 }$ (for
some constant $c$) that vanishes on $S$. Given $n$ lines in $\R^3$, prove that there is a nonzero polynomial of degree $\le c \, n^{ \frac 1 2 }$ (for some other constant $c$) that vanishes on all the lines. Generalize.

\end{enumerate}
\end{document}
