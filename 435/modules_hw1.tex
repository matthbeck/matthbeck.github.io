\documentclass[11pt]{article}
\usepackage{amssymb,amsmath,latexsym,enumerate}

\hoffset=0in 
\voffset=0in
\oddsidemargin=0in
\evensidemargin=0in
\topmargin=-.7in 
%\headsep=0in 
%\headheight=0in
\textwidth=6.5in
\textheight=9in

\newcommand\GL{\operatorname{GL}} 
\newcommand\Inn{\operatorname{Inn}} 
\newcommand\Aut{\operatorname{Aut}} 
\newcommand\Stab{\operatorname{Stab}} 
\newcommand\orb{\operatorname{orb}} 
\newcommand\lcm{\operatorname{lcm}} 
\newcommand\gen[1]{\left< #1 \right>}
\def\Z{\mathbb{Z}}
\def\R{\mathbb{R}}
\def\F{\mathbb{F}}
\def\C{\mathbb{C}}
\def\Q{\mathbb{Q}}

\begin{document}
\setlength{\parindent}{0pt}
\setlength{\parskip}{0.4cm}

\thispagestyle{empty} 

\begin{center} {\bf MATH 435/735 \qquad \qquad Module Homework I} \end{center} 

\begin{enumerate}[(1)]

\item Suppose $R$ is a ring. Prove that is $M$ and $N$ are free $R$-modules, then $M \oplus N$ is also a free $R$-module.

\item Suppose $R$ is a ring and $M$ is an $R$-module. Show that a set that contains a torsion element of $M$ cannot be a basis of $M$.

\item Let $R$ be a commutative ring, viewed as an $R$-module, and let $I$ be an ideal of $R$. Show that any two elements in $I$ are linearly dependent. 

\item (Grad students)
Let $R$ be a commutative ring. Prove that $R^m \cong R^n$ if and only if $m=n$.
(\emph{Hint:} try the same trick as in Monday's lecture---start with a maximal ideal of $R$ to reduce the problem to that of vector spaces.)

\item 
Consider the $\Z$-module $M := \Z \times \Z \times \cdots$
(which you may think of as consisting of all integer sequences).
Let $R$ be the set of all homomorphisms $M \to M$.
This set $R$ becomes a (non-commutative) ring, as usual, by defining for $f, g \in R$
\[
  (f+g)(x) := f(x) + g(x)
  \qquad \text{ and } \qquad
  (fg)(x) := f(g(x)) \, .
\]
Now consider $R$ as an $R$-module.
\begin{enumerate}[(a)]
  \item Define the functions $f_1, f_2, g_1, g_2 \in R$ by
\begin{align*}
  f_1(x_1, x_2, \dots) &:= (x_1, x_3, x_5, \dots) \\
  f_2(x_1, x_2, \dots) &:= (x_2, x_4, x_6, \dots) \\
  g_1(x_1, x_2, \dots) &:= (x_1, 0, x_2, 0, x_3, \dots) \\
  g_2(x_1, x_2, \dots) &:= (0, x_1, 0, x_2, 0, x_3, \dots) \, .
\end{align*}
  Show that $f_1 g_1 = f_2 g_2 = 1$, $f_1 g_2 = f_2 g_1 = 0$, and $g_1 f_1 + g_2 f_2 = 1$.
  \item Show that $\{ f_1, f_2 \}$ is a basis of $R$.
  \item Show that $R \cong R^2$.
\end{enumerate}
(This gives an example that free modules over non-commutative rings need not have a well-defined notion of rank/dimension.)

\end{enumerate}

\end{document}
\end{document}
