\documentclass[11pt]{article} 
\usepackage{amssymb,latexsym,amsmath,enumerate,graphicx}
\usepackage[sc]{mathpazo}
\usepackage[T1]{fontenc}

\hoffset=0in 
\voffset=-.7in
\oddsidemargin=0in
\evensidemargin=0in
\topmargin=0in 
\textwidth=6.5in
\textheight=9in

\def\Z{\mathbb{Z}}
\def\sage{{\tt sage}}

\pagestyle{empty}
\begin{document}
\setlength{\parindent}{0pt}
\setlength{\parskip}{0.2cm}

{\sc 2018 Math 310: Number Theory}
\hfill
{\tt math.sfsu.edu/beck/310.php}

\vspace{.3in}

\begin{center}
\Large{Worksheet 8: Primitive Roots}
\end{center}

\begin{enumerate}

\item Compute all primitive roots mod 6, 7, and 8.

\item Suppose $a$ has order $n$ mod $m$, and $a^k \equiv 1 \bmod m$. Show that~$n|k$.

\item Show that, if $a$ is a primitive root mod $m$, then $\{ a, a^2, \dots, a^{ \phi(m) } \}$ is a
reduced residue system mod~$m$.

\item Suppose $a$ has order $n$ mod $m$, and $\gcd(k,n) = g$. Show that $a^k$ has order $\frac n g$ mod~$m$.
Conclude that this implies the following two corollaries:
  \begin{enumerate}
  \item If $a$ is a primitive root mod $m$, then $a^k$ is also a primitive root mod $m$ if and only
if $\gcd(k, \phi(m)) = 1$.
  \item If there exists a primitive root mod $m$, then there are precisely $\phi(\phi(m))$ primitive
roots.
  \end{enumerate}

\item Andrews 7.1.6, 7.2.15, Stein 2.8, 2.30. % something's wrong with the Stein assignments...
 
\item Write down a precise statement for each definition we have given this week.
For each definition, give an example and a non-example.

\end{enumerate}

\end{document}

