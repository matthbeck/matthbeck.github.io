\documentclass[11pt]{article} 
\usepackage{amssymb,latexsym,amsmath,enumerate,graphicx}
\usepackage[sc]{mathpazo}
\usepackage[T1]{fontenc}

\hoffset=0in 
\voffset=-.7in
\oddsidemargin=0in
\evensidemargin=0in
\topmargin=0in 
\textwidth=6.5in
\textheight=9in

\def\Z{\mathbb{Z}}
\def\sage{{\tt sage}}

\pagestyle{empty}
\begin{document}
\setlength{\parindent}{0pt}
\setlength{\parskip}{0.2cm}

{\sc 2022 Math 310: Number Theory} \hfill {\tt https://matthbeck.github.io/310.html}

\vspace{.3in}

\begin{center}
\Large{Worksheet 3: Modular Arithmetic}
\end{center}

\begin{enumerate}

\item Fix a positive integer $m$, and define the relation $x \sim y$ by $x \equiv y \bmod m$.
Prove that $\sim$ is an equivalence relation.

\item Let $a,b,c,m \in \Z$ with $m > 0$.
  \begin{enumerate}
  \item Show that, if $\gcd(c,m) = 1$, then
  \[
    ac \equiv bc \pmod m
    \qquad \text{ implies } \qquad
    a \equiv b  \pmod m \, .
  \]
  \item Give an example that shows that the gcd condition is necessary.
  \end{enumerate}

\item Suppose $a,b,m \in \Z$ with $m > 0$, and let $g := \gcd(a,m)$. Prove:
  \begin{enumerate}
  \item If $g \nmid b$ then $ax \equiv b \bmod m$ has no solution $x \in \Z$.
  \item If $g \mid b$ then $ax \equiv b \bmod m$ has $g$ distinct solutions $x$ modulo~$m$.
  \item If $g = 1$ then $a$ has a multiplicative inverse modulo~$m$.
  \end{enumerate}

\item Suppose $a,m \in \Z$ with $m > 0$ and $\gcd(a,m) = 1$, and let $\{ r_1, r_2, \dots, r_{ \phi(m) } \}$ be a
reduced residue system modulo~$m$.
  \begin{enumerate}
  \item Show that $\{ a r_1, a r_2, \dots, a r_{ \phi(m) } \}$ is also a reduced residue system modulo~$m$.
  \item Conclude that $r_1 r_2 \cdots r_{ \phi(m) } \equiv (a r_1) (a r_2) \cdots (a r_{ \phi(m) }) \bmod m$ and,
consequently, that
  \[
    a^{ \phi(m) } \equiv 1 \pmod m \, .
  \]
  (This is \emph{Euler's theorem}.)
  \item Prove that, if $p$ is prime and $a \in \Z$, then $a^p \equiv a \bmod p$.
  (This is \emph{Fermat's little theorem}.)
  \item Conclude that, if $p$ is prime and $a,b \in \Z$, then $(a+b)^p \equiv a^p + b^p \bmod p$.
  (This is every freshman's dream.)
  \end{enumerate}

\item Suppose $p$ is prime.
Prove that $x^2 \equiv 1 \bmod p$ has precisely the two solutions $x \equiv \pm 1 \bmod p$.

\item Suppose $m \in \Z_{ >0 }$.
  \begin{enumerate}
  \item Show that, if $m>4$ is not prime, then $(m-1)! \equiv 0 \bmod m$.
  \item Now suppose $m$ is prime. Show that if $a \not\equiv 0, \pm 1 \bmod m$ then there exists $b \not\equiv 0, \pm 1, a \bmod m$ such that $ab \equiv 1 \bmod m$.
  \item Conclude \emph{Wilson's theorem}: $(m-1)! \equiv -1 \bmod m$ if and only if $m$ is prime.
  \end{enumerate}

\item Andrews 5.1.1--3, 3.2.3, 5.2.3, and 5.2.19.

\item Experiment with the \sage \ command {\tt mod}.
Compare the running times of $2^{1000000000000} \bmod 3$ and $(2 \bmod 3)^{1000000000000}$.
What do you think \sage \ does?

\item Compute $7^{ 43 } \bmod 11$ without \sage. Check your answer with \sage.

\item Write down a precise statement for each definition we have given this week.
For each definition, give an example and a non-example.

\end{enumerate}

\end{document}

