\documentclass[11pt]{article} 
\usepackage{amssymb,latexsym,amsmath,enumerate,graphicx}
\usepackage[sc]{mathpazo}
\usepackage[T1]{fontenc}

\hoffset=0in 
\voffset=-.7in
\oddsidemargin=0in
\evensidemargin=0in
\topmargin=0in 
\textwidth=6.5in
\textheight=9in

\def\Z{\mathbb{Z}}
\def\sage{{\tt sage}}
\newcommand\leg[2]{\left( \frac{ #1 }{ #2 } \right)}

\pagestyle{empty}
\begin{document}
\setlength{\parindent}{0pt}
\setlength{\parskip}{0.2cm}

{\sc 2018 Math 310: Number Theory}
\hfill
{\tt math.sfsu.edu/beck/310.php}

\vspace{.3in}

\begin{center}
\Large{Worksheet 10: Quadratic Residues}
\end{center}

\begin{enumerate}

\item Make a list of all quadratic residues mod 2, 3, 5, 7, and 11.

\item In this exercise, we'll prove another one of Euler's theorems:
If $p$ is an odd prime, then $a$ is a quadratic residue mod $p$ if and only if $a^{ \frac{ p-1 }{ 2 } } \equiv 1 \bmod p$.

  \begin{enumerate}

  \item Prove the ``$\Longrightarrow$'' direction, e.g., by recalling another theorem by Euler.

  \item For the ``$\Longleftarrow$'' direction, you may assume that there exits a primitive root $r$ mod $p$ (which is true, although we haven't prove it).
  Assuming $a^{ \frac{ p-1 }{ 2 } } \equiv 1 \bmod p$, use the fact that $a \equiv r^n$ for some $n$, and show that $n$ is even.

  \end{enumerate}

\item Use Euler's theorem to prove, given a primitive root $r$ mod $p$ (as above, an odd prime), that $g^n$ is a quadratic residue mod $p$ if and only if $n$ is even.
Conclude that, for an odd prime $p$, exactly half the integers between 1 and $p-1$ are quadratic residues mod~$p$.

\item Let $p$ be and odd prime not dividing $a$ and $b$. Show that:

  \begin{enumerate}

  \item $\leg{ab} p = \leg a p \leg b p$

  \item $\leg a p \equiv a^{ \frac{ p-1 }{ 2 } } \bmod p$
 
  \end{enumerate}

\item Andrews 9.2.2. 
 
\item Write down a precise statement for each definition we have given this week.
For each definition, give an example and a non-example.

\end{enumerate}

\end{document}

