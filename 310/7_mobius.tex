\documentclass[11pt]{article} 
\usepackage{amssymb,latexsym,amsmath,enumerate,graphicx}
\usepackage[sc]{mathpazo}
\usepackage[T1]{fontenc}

\hoffset=0in 
\voffset=-.7in
\oddsidemargin=0in
\evensidemargin=0in
\topmargin=0in 
\textwidth=6.5in
\textheight=9in

\def\Z{\mathbb{Z}}
\def\sage{{\tt sage}}

\pagestyle{empty}
\begin{document}
\setlength{\parindent}{0pt}
\setlength{\parskip}{0.2cm}

{\sc 2022 Math 310: Number Theory} \hfill {\tt https://matthbeck.github.io/310.html}

\vspace{.3in}

\begin{center}
\Large{Worksheet 7: The M\"obius Function}
\end{center}

\begin{enumerate}

\item Show that $\mu(n)$ is multiplicative.

\item Prove that
\[
  \sum_{ d|n } \mu(d) \ = \ \begin{cases}
    1 & \text{ if } n=1, \\
    0 & \text{ if } n>1.
  \end{cases}
\]
\emph{Hint:} for $n>1$, try induction on the number of prime factors of~$n$.

\item Suppose that $g(n)$ is multiplicative, and let
\[
  f(n) \ := \ \sum_{ d|n } g(d) \, .
\]
Prove that $f(n)$ is also multiplicative.

\item Prove the \emph{M\"obius Inversion Formula}:
\[
  f(n) \ = \ \sum_{ d|n } g(d)
  \qquad \text{ if and only if } \qquad
  g(n) \ = \ \sum_{ d|n } \mu(d) \, f(\tfrac n d) \, .
\]
\emph{Hint:} write sums like the one on the right-hand side as
\[
  \sum_{ d|n } \mu(d) \, f(\tfrac n d) \ = \ \sum_{ de = n } \mu(d) \, f(e) \, .
\]

\item Andrews 6.4.1, 6.4.3, 6.4.7, 6.4.8, 6.4.11.
 
\item Write down a precise statement for each definition we have given this week.
For each definition, give an example and a non-example.

\end{enumerate}

\end{document}

