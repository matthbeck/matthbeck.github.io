\documentclass[11pt]{article} 
\usepackage{amssymb,latexsym,amsmath,enumerate,graphicx}
\usepackage[sc]{mathpazo}
\usepackage[T1]{fontenc}

\hoffset=0in 
\voffset=-.7in
\oddsidemargin=0in
\evensidemargin=0in
\topmargin=0in 
\textwidth=6.5in
\textheight=9in

\def\Z{\mathbb{Z}}
\def\sage{{\tt sage}}

\pagestyle{empty}
\begin{document}
\setlength{\parindent}{0pt}
\setlength{\parskip}{0.2cm}

{\sc 2018 Math 310: Number Theory}
\hfill
{\tt math.sfsu.edu/beck/310.php}

\vspace{.3in}

\begin{center}
\Large{Worksheet 6: Arithmetic Functions}
\end{center}

\begin{enumerate}

\item Let $p, q$ be two distinct primes. Compute $\phi(pq)$.

\item Let $p$ be prime and $k \in \Z_{ >0 }$. Compute
\begin{enumerate}
  \item $\phi(p^k)$
  \item $\tau(p^k)$
  \item $\sigma(p^k)$
\end{enumerate} 

\item Prove that $\tau$ and $\sigma$ are \emph{multiplicative}, that is, $\tau(mn) = \tau(m) \, \tau(n)$ and
$\sigma(mn) = \sigma(m) \, \sigma(n)$ whenever $\gcd(m,n) = 1$.
(\emph{Hint:} start with the case $m = p^j$, $n = q^k$ for distinct primes $p$ and~$q$.)

\item Fix $m, n \in \Z_{ >0 }$ with $\gcd(m,n) = 1$.
Consider the function $f: \Z_{ mn }^* \to \Z_m^* \times \Z_n^*$ given by
\[
  f(k) := (k \bmod m, \ k \bmod n) \, .
\]
\begin{enumerate}
  \item Show that $f$ is well defined.
  \item Show that $f$ is one-to-one.
  \item Show that $f$ is onto. (\emph{Hint:} Chinese Remainder Theorem.)
  \item Conclude that $\phi(mn) = \phi(m) \, \phi(n)$.
\end{enumerate} 

\item Derive formulas for $\phi(n)$, $\tau(n)$, and $\sigma(n)$ in terms of the prime factorization of~$n$.

\item Fix $n \in \Z_{ >0 }$, and for $d|n$, let
\[
  S_d \ := \ \left\{ m \in [n] \, : \, \gcd(m,n) = d \right\} .
\]
\begin{enumerate}
  \item Come up with a bijection $S_d \to \Z_{ \frac n d }^*$.
  \item Convince yourself that
  \[
    [n] \ = \ \bigcup_{ d|n } S_d
  \]
  as a disjoint union, and conclude that
  \[
    \sum_{ d|n } \phi(d) \ = \ n \, .
  \]
\end{enumerate} 

\item Andrews 6.1.1, 6.1.4, 6.2.2, 6.2.9.
 
\item Write down a precise statement for each definition we have given this week.
For each definition, give an example and a non-example.

\end{enumerate}

\end{document}

