\documentclass[11pt]{article} 
\usepackage{amssymb,latexsym,amsmath,enumerate,graphicx}
\usepackage[sc]{mathpazo}
\usepackage[T1]{fontenc}

\hoffset=0in 
\voffset=-.7in
\oddsidemargin=0in
\evensidemargin=0in
\topmargin=0in 
\textwidth=6.5in
\textheight=9in

\def\Z{\mathbb{Z}}
\def\sage{{\tt sage}}

\pagestyle{empty}
\begin{document}
\setlength{\parindent}{0pt}
\setlength{\parskip}{0.2cm}

{\sc 2018 Math 310: Number Theory}
\hfill
{\tt math.sfsu.edu/beck/310.php}

\vspace{.3in}

\begin{center}
\Large{Worksheet 4: Chinese Remainder Theorem}
\end{center}

\begin{enumerate}

\item Let $a$ by the day (of the month) you were born and $b$ the month.\footnote{
If $a = 26$ and $b = 9$... happy birthday!!
}
Find $x \in \Z$ such that
\[
  x \equiv a \bmod 31
  \qquad \text{ and } \qquad
  x \equiv b \bmod 12 \, .
\]

\item\label{crt2eq}
Suppose $m, n \in \Z_{ >0 }$ are relatively prime, and $a, b \in \Z$.
Prove that
\[
  x \equiv a \bmod m
  \qquad \text{ and } \qquad
  x \equiv b \bmod n 
\]
has a solution $x \in \Z$ and that $x$ is unique modulo~$mn$.

\item Generalize the statement (and your proof) of \ref{crt2eq}.\ to a system of $k$ congruences.

\item Andrews 5.3.1. (Feel free to use \sage.)

\item Write down a precise statement for each definition we have given this week.
For each definition, give an example and a non-example.

\end{enumerate}

\end{document}

