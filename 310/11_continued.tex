\documentclass[11pt]{article} 
\usepackage{amssymb,latexsym,amsmath,enumerate,graphicx,multicol,multirow}
\usepackage[sc]{mathpazo}
\usepackage[T1]{fontenc}

\hoffset=0in 
\voffset=-.7in
\oddsidemargin=0in
\evensidemargin=0in
\topmargin=0in 
\textwidth=6.5in
\textheight=9in

\def\Q{\mathbb{Q}}
\def\R{\mathbb{R}}
\def\Z{\mathbb{Z}}
\def\sage{{\tt sage}}
\newcommand\leg[2]{\left( \frac{ #1 }{ #2 } \right)}

\pagestyle{empty}
\begin{document}
\setlength{\parindent}{0pt}
\setlength{\parskip}{0.2cm}

{\sc 2018 Math 310: Number Theory}
\hfill
{\tt math.sfsu.edu/beck/310.php}

\vspace{.3in}

\begin{center}
\Large{Worksheet 11: Continued Fractions}
\end{center}

\begin{enumerate}

\item Find continued fraction expansions for
  \begin{multicols}{3}
  \begin{enumerate}
  \item $\frac{ 100 }{ 37 }$
  \item $\frac{ 1001 }{ 45 }$
  \item $\frac{ 21 }{ 13 }$
  \item $\frac{ 1000 }{ 301 }$
  \item $\frac{ 13 }{ 35 }$
  \item $\frac{ \sqrt 5 - 1 }{ 2 }$
  \end{enumerate}
  \end{multicols}

\item Compute
  \begin{multicols}{3}
  \begin{enumerate}
  \item $[2,3,2,3, \dots]$
  \item $[1,2,1,2, \dots]$
  \item $[1,2,2,2, \dots]$
  \end{enumerate}
  \end{multicols}

\item Compute the start of a continued fraction expansion for $\pi$ and compare the accuracy of $[a_0, a_1,
\dots, a_n]$ with that of the decimal expansion up to $n$ digits, for $n = 1, 2, 3, 4$.

\item Compute the continued fraction expansions of $e$, $\sqrt{19}$, and $\tan(1)$ with \sage.

\item\label{two} Prove that $[a_0, a_1, \dots, a_n] = \frac{ p_n }{ q_n }$ where

  \begin{tabular}{llll}
    $p_0 = a_0$ \hspace{1cm} & $p_1 = a_1 \, p_0 + 1$ \hspace{1cm} & $p_n = a_n \, p_{ n-1 } + p_{ n-2 }$ & \multirow{2}{*}{for $n \ge 2$} \\
    $q_0 = 1$                & $q_1 = a_1$                         & $q_n = a_n \, q_{ n-1 } + q_{ n-2 }$ 
  \end{tabular}

\item Keeping the notation from \ref{two}., show that
\[
  p_n \, q_{ n-1 } - q_n \, p_{ n-1 } \ = \ (-1)^{ n-1 } .
\]
(\emph{Hint:} an easy way to proceed is to extend the definition of the $p_j$'s and $q_j$'s by setting $p_{ -2 }
= q_{ -1 } = 0$ and $q_{ -2 } = p_{ -1 } = 1$.)
Conclude that $\gcd(p_n, q_n) = 1$, i.e., the fraction $[a_0, a_1, \dots, a_n] = \frac{ p_n }{ q_n }$ is written
in lowest terms.

\item Prove that the sequence $( \frac{ p_n }{ q_n } )_{ n \ge 1 }$ converges.
(\emph{Hint:} first show that $(q_n)_{ n \ge 1 }$ is strictly increasing, and then prove that $( \frac{ p_n }{ q_n } )_{ n \ge 1 }$ is a Cauchy sequence.)

\item Show that
\[
  p_n \, q_{ n-2 } - q_n \, p_{ n-2 } \ = \ (-1)^n \, a_n
\]
and conclude that $( \frac{ p_{ 2n } }{ q_{ 2n } } )_{ n \ge 1 }$ increases and $( \frac{ p_{ 2n+1 } }{ q_{ 2n+1 } } )_{ n \ge 1 }$ decreases.

\item Suppose $a \in \R \setminus \Q$ has an eventually periodic continued fraction expansion, i.e.,
\[
  a \ = \ [a_0, a_1, \dots, a_n, a_{ n+1 } , \dots, a_{ n+k }, a_{ n+1 } , \dots, a_{ n+k }, \dots ]
\]
for some positive integers $n$ and $k$. Prove that $[a_{ n+1 } , \dots, a_{ n+k }, a_{ n+1 } , \dots, a_{ n+k },
\dots ]$ satisfies a quadratic equation and conclude that $a = b+cx$ where $b, c \in \Q$ and $x \in \R$
satisfies a quadratic equation, i.e., $a$ is a \emph{quadratic irrational}.

\item Stein 5.1--4. 
 
\item Write down a precise statement for each definition we have given this week.
For each definition, give an example and a non-example.

\end{enumerate}

\end{document}

