\documentclass[11pt]{article} 
\usepackage{amssymb,latexsym,enumerate,graphicx}
\usepackage[sc]{mathpazo}
\usepackage[T1]{fontenc}

\hoffset=0in 
\voffset=0in
\oddsidemargin=0in
\evensidemargin=0in
\topmargin=0in 
\textwidth=6.5in
\textheight=8.6in

\def\Z{\mathbb{Z}}
\def\sage{{\tt sage} }

\pagestyle{empty}

\begin{document}
\setlength{\parindent}{0pt}
\setlength{\parskip}{0.2cm}

$\mbox{}$
\vspace{-1in}

{\sc 2022 Math 310: Number Theory} \hfill {\tt https://matthbeck.github.io/310.html}

\vspace{.1in}

\begin{center}
\Large{Worksheet 2: Primes}
\end{center}

\begin{enumerate}

\item Let $a, b \in \Z_{ >0 }$.
Show that, if $g = \gcd(a,b)$ then $\gcd(\frac a g, \frac b g) = 1$.

\item Give a careful definition of a \emph{prime number}.

\item Let $a, b, c \in \Z_{ >0 }$.
  \begin{enumerate}
  \item Prove that, if $a \mid bc$ and $\gcd(a,b) = 1$, then $a \mid c$.
  \item Conclude that if $p$ is prime and $p \mid ab$, then $p \mid a$ or $p \mid b$.
  \item Give a counterexample that shows the previous sentence is wrong if $p$ is not prime.
  \end{enumerate}

\item Prove the \emph{Fundamental Theorem of Arithmetic}: for every integer $n \ge 2$ there exist unique primes
$p_1, p_2, \dots, p_k$ and positive integers $a_1, a_2, \dots, a_k$ such that
\[
  n \ = \ p_1^{ a_1 } p_2^{ a_2 } \cdots p_k^{ a_k } .
\]
  \begin{enumerate}
  \item For existence, try induction on~$n$.
  \item For uniqueness, you may use 3(b).
  \end{enumerate}

\item Andrews 2.4.5 \& 6.

\item Experiment with the \sage commands {\tt factor} and {\tt is\underline{ }prime}.
Try them with a 100-digit number and a 150-digit number and compare the four running times (e.g., by using {\tt
\%time} before the command). What's going on here?

\item \emph{Preview: Clock Arithmetic.}
% In this clock world, we only consider numbers (nonnegative integers) less than the number of hours on the clock.   In this example, we have a 6-hour clock.
The numbers on the 6-hour clock are the
remainders we get when we divide by, in this case, 6.   Adding 3 to 4
gets us to 1, which is also the remainder of dividing 3+4 by 6.

\begin{enumerate}

% \setcounter{enumi}{2}  %% next number will be n+1

\item Explain why the number at the top of the clock is 0 rather
than 6.

\item Complete the clock addition table and this clock multiplication table

\begin{minipage}[t]{0.4\textwidth}
\begin{tabular}{|c||c|c|c|c|c|c|}
\hline
+ & 0 & 1 & 2 &  3  & 4 & 5 \\ \hline \hline
0 & \hspace*{.2in} & \hspace*{.2in} & \hspace*{.2in} &
\hspace*{.2in} & \hspace*{.2in} & \hspace*{.2in}\\
  &  &  &  &  &  &  \\ \hline
1 &  &  &  &  &  & \\
  &  &  &  &  &  & \\ \hline
2 &  &  &  &  &  & \\
  &  &  &  &  &  & \\ \hline
3 &  &  &  &  &  & \\
  &  &  &  &  &  & \\ \hline
4 &  &  &  &  &  & \\
  &  &  &  &  &  & \\ \hline
5 &  &  &  &  &  & \\
  &  &  &  &  &  & \\ \hline
\end{tabular}
\end{minipage}
\begin{minipage}[t]{0.2\textwidth}
\includegraphics[width=.8\textwidth]{6-hour-clock.png}
\end{minipage}
\begin{minipage}[t]{0.4\textwidth}
\begin{tabular}{|c||c|c|c|c|c|c|}
\hline
$\cdot$ & 0 & 1 & 2 &  3  & 4 & 5 \\ \hline \hline
0 & \hspace*{.2in} & \hspace*{.2in} & \hspace*{.2in} &
\hspace*{.2in} & \hspace*{.2in} & \hspace*{.2in}\\
  &  &  &  &  &  &  \\ \hline
1 &  &  &  &  &  & \\
  &  &  &  &  &  & \\ \hline
2 &  &  &  &  &  & \\
  &  &  &  &  &  & \\ \hline
3 &  &  &  &  &  & \\
  &  &  &  &  &  & \\ \hline
4 &  &  &  &  &  & \\
  &  &  &  &  &  & \\ \hline
5 &  &  &  &  &  & \\
  &  &  &  &  &  & \\ \hline
\end{tabular}
\end{minipage}

\vspace*{.2in}

\item What patterns do you see in these two tables?

\end{enumerate}

\item Write down a precise statement for each definition we have given this week.
For each definition, give an example and a non-example.

\end{enumerate}

\end{document}

