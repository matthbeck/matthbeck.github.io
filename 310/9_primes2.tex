\documentclass[11pt]{article} 
\usepackage{amssymb,latexsym,amsmath,enumerate,graphicx}
\usepackage[sc]{mathpazo}
\usepackage[T1]{fontenc}

\hoffset=0in 
\voffset=-.7in
\oddsidemargin=0in
\evensidemargin=0in
\topmargin=0in 
\textwidth=6.5in
\textheight=9in

\def\Z{\mathbb{Z}}
\def\sage{{\tt sage}}

\pagestyle{empty}
\begin{document}
\setlength{\parindent}{0pt}
\setlength{\parskip}{0.2cm}

{\sc 2018 Math 310: Number Theory}
\hfill
{\tt math.sfsu.edu/beck/310.php}

\vspace{.3in}

\begin{center}
\Large{Worksheet 9: Primes Again}
\end{center}

\begin{enumerate}

\item Prove that there are infinitely many primes.
One way to do this is by means of contradiction: assuming $p_1, p_2, \dots, p_k$ are the only primes,
consider the number $p_1 p_2 \cdots p_k + 1$.\footnote{
Your proof probably uses two ``obvious'' but nontrivial facts, namely, (1) that every integer can be uniquely
factored into primes, and (2) that two adjacent integers are relatively prime.
}

\item Prove that there are infinitely primes $\equiv 3 \bmod 4$.
(\emph{Hint:} assuming $p_1, p_2, \dots, p_k$ are the only primes $\equiv 3 \bmod 4$, consider the number $4 p_1 p_2 \cdots p_k - 1$.)
Explain why this is much easier than to prove that there are infinitely primes $\equiv 1 \bmod 4$.\footnote{
There is a general result, known as \emph{Dirichlet's Theorem}, which says that given $a,b \in \Z$ with $\gcd(a,b) = 1$, there are infinitely primes $\equiv a \bmod b$.
}

\item Show that if $n$ is composite, then so is $2^n - 1$.
Thus the \emph{Mersenne number} $M_p := 2^p - 1$ can only possibly be prime if $p$ is prime.
Find the first five Mersenne primes, and the first five composite Mersenne numbers $M_p$ for which $p$ is
prime.

\item Prove that $p \in \Z_{ >1 }$ is prime if and only if $a^{ p-1 } \equiv 1 \bmod p$ for all $a
\not\equiv 0 \bmod p$.\footnote{
The second condition is subtly different from that of a \emph{Carmichael number}, which is a composite number $n$ such that $a^{ n-1 } \equiv 1 \bmod n$ whenever $\gcd(a,n) = 1$.
}
Explain how this can be used for a test for \emph{compositeness} of an integer without actually factoring
it.
 
% \item Andrews 
 
\item Write down a precise statement for each definition we have given this week.
For each definition, give an example and a non-example.

\end{enumerate}

\end{document}

